\documentclass[12pt, a4paper, oneside]{ctexbook}
\usepackage{amsmath, amsthm, amssymb, bm, graphicx, hyperref, mathrsfs}

\title{{\Huge{\textbf{尸变函数抄书笔记}}}\\———主要参考Folland}
\author{Joker}
\date{\today}
\linespread{1.5}
\newtheorem{theorem}{定理}[section]
\newtheorem{definition}[theorem]{定义}
\newtheorem{lemma}[theorem]{引理}
\newtheorem{corollary}[theorem]{推论}
\newtheorem{example}[theorem]{例}
\newtheorem{proposition}[theorem]{命题}

\begin{document}
这一节建立了用来构造测度的工具
\begin{definition}
    非空集合$X$上的外测度是函数$\mu^*:\mathcal{P}(X)\rightarrow [0,+\infty]$,并且满足以下条件:\\
    1. $\mu^*(\phi)=0$\\
    2. $A\subset B\Rightarrow \mu^*(A)\leq \mu^*(B)$\\
    3. $\mu^*(\bigcup_1^{\infty}A_j)\leq \sum_1^{\infty}\mu^*(A_j)$
\end{definition}
\begin{proposition}
    设$\varepsilon\subset\mathcal{P}(X),\rho:\varepsilon\rightarrow [0,+\infty]$满足:$\phi\in\varepsilon,X\in\varepsilon,\rho(\phi)=0$\\
    对任意$A\in\mathcal{P}(X)$,定义:\\
    $\mu^*(A)=\inf\left\{\sum_1^{\infty}\rho(E_j):E_j\in\varepsilon,A\subset\bigcup_1^{\infty}E_j\right\}$\\
    则$\mu^*$是一个外测度
    \begin{proof}
        首先,由于$X\in\varepsilon$,故$\forall A\in\mathcal{P}(X),A\subset X$,该定义是良定义的\\
        1. $\phi\in\varepsilon,\mu^*(\phi)=\rho(\phi)=0$\\
        2. $A\subset B$,$\forall\left\{E_j\right\}_1^{\infty}$覆盖$B$,也覆盖$A$,即$\mu^*(A)\leq\sum_1^{\infty}\rho(E_j)$\\
        对任意$\left\{E_j\right\}_1^{\infty}$成立,于是$\mu^*(A)\leq\mu^*(B)$\\
        3. $\forall \epsilon>0$对于$A_j\in\mathcal{P}(X),\exists\left\{E_j^k\right\}_{k=1}^{\infty}\subset\varepsilon:\sum_{k=1}^{\infty}\rho(E_j^k)<\mu^*(A_j)+\epsilon2^{-j}$\\
        令$A=\bigcup_1^{\infty}A_j\subset\bigcup_{j,k}E_j^k,\mu^*(A)\leq\sum_{j,k}\rho(E_j^k)\leq\sum_1^{\infty}\mu^*(A_j)+\epsilon$\\
        由$\epsilon$任意性,$\mu^*(\bigcup_1^{\infty}A_j)\leq\sum_1^{\infty}\mu^*(A_j)$
    \end{proof}
\end{proposition}
\begin{definition}
    设$\mu^*$为$X$上的一个外测度,$A\subset X$称为$\mu^*$-可测的,若:\\
    $\mu^*(E)=\mu^*(E\cap A)+\mu^*(E\cap A^c)$对所有$E\subset X$成立
\end{definition}
由于$\mu^*(E)\leq\mu^*(E\cap A)+\mu^*(E\cap A^c)$平凡成立,故$A$是$\mu^*$-可测的,只
需验证$\mu^*(E)\geq\mu^*(E\cap A)+\mu^*(E\cap A^c)$对所有$E:\mu^*(E)<+\infty$成立
\begin{theorem}
    Caratheodory's Theorem\\
    若$\mu^*$是$X$上的一个外测度,所有$\mu^*$-可测集构成的集合$\,\mathcal{M}$是$\sigma$-代数,$\mu^*$在$\mathcal{M}$上
    的限制是一个完全的测度
    \begin{proof}
        首先证明$\mathcal{M}$是$\sigma$-代数\\
        由于$\mu^*$-可测的定义关于$A,A^c$对称,故$\mathcal{M}$关于取补集封闭\\
        设$A,B\in\mathcal{M}$,$\mu^*(E)=\mu^*(E\cap A)+\mu^*(E\cap A^c)$\\
        $=\mu^*(E\cap A\cap B)+\mu^*(E\cap A\cap B^c)+\mu^*(E\cap A^c\cap B)+\mu^*(E\cap A^c\cap B^c)$\\
        $A\cup B=(A\cap B)\cup(A\cap B^c)\cup(B\cap A^c)$\\
        $\mu^*(E)\geq\mu^*(E\cap(A\cup B))+\mu^*(E\cap(A\cup B)^c)$\\
        从而$A\cup B$可测,$A\cup B\in\mathcal{M}$\\
        若还有$A\cap B=\phi$,\\
        $\mu^*(A\cup B)=\mu^*((A\cup B)\cap A)+\mu^*((A\cup B)\cap A^c)=\mu^*(A)+\mu^*(B)$\\
        即$\mu^*$在$\mathcal{M}$上有限可加\\
        现往证$\mathcal{M}$对于可数不交并封闭\\
        设$\left\{A_j\right\}_1^{\infty}$为$\mathcal{M}$中一列不交集,令$B_n=\bigcup_1^nA_j,B=\bigcup_1^{\infty}A_j$\\
        归纳可得$B_n\in\mathcal{M}$,$\mu^*(E\cap B_n)=\mu^*(E\cap B_n\cap A_n)+\mu^*(E\cap B_n\cap A_n^c)$\\
        $=\mu^*(E\cap A_n)+\mu^*(E\cap B_{n-1})$\\
        令$B_0=\phi$,归纳可得$\mu^*(E\cap B_n)=\sum_1^n\mu^*(E\cap A_j)$\\
        $\mu^*(E)=\mu^*(E\cap B_n)+\mu^*(E\cap B_n^c)\geq\sum_1^n\mu^*(E\cap A_j)+\mu^*(E\cap B)$\\
        令$n\rightarrow\infty$,$\mu^*(E)\geq\sum_1^{\infty}\mu^*(E\cap A_j)+\mu^*(E\cap B)\\
        \geq\mu^*(\bigcup_1^{\infty}(E\cap A_j))+\mu^*(E\cap B^c)=\mu^*(E\cap B)+\mu^*(E\cap B^c)$\\
        于是$B=\bigcup_1^{\infty}A_j\in\mathcal{M}$\\
        在上式中取$E=B$,得$\mu^*(\bigcup_1^{\infty}A_j)=\sum_1^{\infty}\mu^*(A_j)$\\
        于是$\mu^*_{|\mathcal{M}}$是测度\\
        $\mu^*(A)=0\Rightarrow\mu^*(E)\leq\mu^*(E\cap A)+\mu^*(E\cap A^c)=\mu^*(E\cap A^c)\leq\mu^*(E)$\\
        从而$A\in\mathcal{M}$,即$\mu^*_{|\mathcal{M}}$是完全的
    \end{proof}
\end{theorem}
\begin{definition}
    设$\mathcal{A}\subset\mathcal{P}(X)$是一个代数,$\mu_0:\mathcal{A}\rightarrow[0,+\infty]$称为预测度,若其
    满足以下条件:\\
    $\mu_0(\phi)=0$\\
    设$\left\{A_j\right\}_1^{\infty}$是$\mathcal{A}$中的一列不交集,并且$\bigcup_1^{\infty}A_j\in\mathcal{A}$,则$\mu_0(\bigcup_1^{\infty}A_j)=\sum_1^{\infty}\mu_0(A_j)$\\
    从而$\mu_0$也是有限可加的
\end{definition}
若$\mu_0$是$\mathcal{A}\subset\mathcal{P}(X)$是一个预测度,则由命题2.3.2 $\mu_0$可以诱导一个外测度:
$\mu^*(E)=\inf\left\{\sum_1^{\infty}\mu_0(A_j):A_j\in\mathcal{A},E\subset\bigcup_1^{\infty}A_j\right\}$
\begin{proposition}
    设$\mu_0$是$\mathcal{A}$上的一个预测度,$\mu^*$是其诱导的外测度,那么:\\
    a. $\mu^*_{|\mathcal{A}}=\mu_0$;\\
    b. $\mathcal{A}$中的每个集合都是$\mu^*$-可测的
    \begin{proof}
        a. 设$E\in\mathcal{A}$,若$E\subset \bigcup_1^{\infty}A_j,A_j\in\mathcal{A}$,\\
        令$B_n=E\cap(A_n\setminus(\bigcup_1^{n-1}A_j))\in\mathcal{A}$
        于是$B_n$是不交的,且$\bigcup_1^{\infty}B_n=E$\\
        $\mu_0(E)=\sum_1^{\infty}\mu_0(B_n)\leq\sum_1^{\infty}\mu_0(A_n)$\\
        由$A_n$的任意性,$\mu_0(E)\leq\mu^*(E)\leq\mu_0(E)$\\
        其中第二个不等号是明显的,因为$E\subset E\in\mathcal{A}$\\
        b. 设$A\in\mathcal{A},E\in\mathcal{P}(X),\forall \epsilon>0,\exists\left\{B_j\right\}_1^{\infty}\subset\mathcal{A}:$\\
        $A\subset\bigcup_1^{\infty}B_j,\sum_1^{\infty}\mu_0(B_j)<\mu^*(E)+\epsilon$\\
        于是$\mu^*(E)+\epsilon>\sum_1^{\infty}\mu_0(B_j)=\sum_1^{\infty}\mu_0((B_j\cap A)\cup(B_j\cap A^c))$\\
        $=\sum_1^{\infty}\mu_0(B_j\cap A)+\sum_1^{\infty}\mu_0(B_j\cap A^c)$\\
        $\geq\mu^*(E\cap A)+\mu^*(E\cap A^c)$\\
        由$\epsilon$的任意性,$\mu^*(E)\geq\mu^*(E\cap A)+\mu^*(E\cap A^c)$\\
        从而$A$是$\mu^*$-可测的
    \end{proof}
\end{proposition}
\begin{theorem}
    令$\mathcal{A}\subset\mathcal{P}(X)$是一个代数,$\mu_0$是$\mathcal{A}$上的预测度,$\mathcal{M}$是$\mathcal{A}$生成的$\sigma$-代数
    则存在$\mathcal{M}$上的测度$\mu$,其在$\mathcal{A}$上的限制等于$\mu_0$,且$\mu=\mu^*_{|\mathcal{M}}$,其
    中$\mu^*$是$\mu_0$诱导的外测度,若$\nu $是$\mathcal{M}$上的另一个$\mu_0$由扩张的测度,则$\nu(E)\leq\mu(E)$
    $\forall E\in\mathcal{M}$其中等号在$\mu(E)<+\infty$时成立,若$\mu_0$是$\sigma$-有限的,则$\mu$是$\mu_0$在\\
    $\mathcal{M}$上唯一的扩张
    \begin{proof}
        首先由Caratheodory Theorem和命题2.3.6 $\mu_0$可诱导$X$上的外测度$\mu^*$,
        其所有$\mu^*$-可测集构成包含$\mathcal{A}$的$\sigma$-代数,自然也包含$\mathcal{M}$,且$\mu^*$在$\mathcal{M}$上的限制
        是测度,即$\mu=\mu_{|\mathcal{M}}^*$,$\mu_{|\mathcal{A}}=\mu^*_{\mathcal{A}}=\mu_0$\\
        设$E\in\mathcal{M}$,$\left\{A_j\right\}_1^{\infty}$为任意$\mathcal{A}$中覆盖$E$的集列,则\\
        $\nu(E)\leq\nu(\bigcup_1^{\infty}A_j)\leq\sum_1^{\infty}\nu(A_j)=\sum_1^{\infty}\mu_0(A_j)$\\
        $\nu(E)\leq\mu^*_{|\mathcal{M}}(E)=\mu(E)$\\
        若$\mu(E)<+\infty$,$\forall\epsilon>0,\exists\left\{A_j\right\}_1^{\infty}\subset\mathcal{A}:$\\
        $\mu(A)\leq\sum_1^{\infty}\mu(A_j)=\sum_1^{\infty}\mu_0(A_j)<\mu(E)+\epsilon$\\
        $\mu(A\setminus E)=\mu(A)-\mu(E)<\epsilon$\\
        $\mu(E)\leq\mu(A)=\lim_{n \to \infty}\mu(\bigcup_1^nA_j)=\lim_{n \to \infty}\nu(\bigcup_1^nA_j)\\
        =\nu(A)=\nu(A\setminus E)+\nu(E)\leq\mu(A\setminus E)+\nu(E)<\epsilon+\nu(E)$\\
        由$\epsilon$任意性知$\nu(E)\leq\mu(E)$\\
        设$\mu_0$是$\sigma$-有限的,即$\exists\left\{A_j\right\}_1^{\infty}\subset\mathcal{A}:X=\bigcup_1^{\infty}A_j,\mu_0(A_j)<+\infty$\\
        不妨设$A_n$是不交的,否则代之以$B_n=A_n\setminus(\bigcup_1^{n-1}A_j)$\\
        $\forall E\in\mathcal{M},\mu(E)=\mu(E\cap X)=\mu(E\cap(\bigcup_1^{\infty}A_j))=\mu(\bigcup_1^{\infty}(E\cap A_j))$\\
        $=\lim_{n\to\infty}\mu(\bigcup_1^n(E\cap A_j))=\lim_{n\to\infty}\sum_1^n\mu(E\cap A_j)$\\
        $=\lim_{n\to\infty}\sum_1^n\nu(E\cap A_j)=\lim_{n\to\infty}\nu(\bigcup_1^n(E\cap A_j))$\\
        $=\nu(\bigcup_1^{\infty}(E\cap A_j))=\nu(E\cap X)=\nu(E)$\\
        于是$\mu=\nu$
    \end{proof}
\end{theorem}
\end{document}
