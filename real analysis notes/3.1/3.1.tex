\documentclass[12pt, a4paper, oneside]{ctexbook}
\usepackage{amsmath, amsthm, amssymb, bm, graphicx, hyperref, mathrsfs}

\title{{\Huge{\textbf{尸变函数抄书笔记}}}\\———主要参考Folland}
\author{Joker}
\date{\today}
\linespread{1.5}
\newtheorem{theorem}{定理}[section]
\newtheorem{definition}[theorem]{定义}
\newtheorem{lemma}[theorem]{引理}
\newtheorem{corollary}[theorem]{推论}
\newtheorem{example}[theorem]{例}
\newtheorem{proposition}[theorem]{命题}

\begin{document}
$f:X\to Y$可诱导$f^{-1}:\mathcal{P}(Y)\to\mathcal{P}(X),f^{-1}(E)=\left\{x\in X:f(x)\in E\right\}$\\
且$f^{-1}$与并,交,补可换序。从而,$\mathcal{N}\subset\mathcal{P}(Y)$为$\sigma$-代数$\Rightarrow\left\{f^{-1}(E):E\in\mathcal{N}\right\}$
为$X$上的$\sigma$-代数
\begin{definition}
    设$(X,\mathcal{M}),(Y,\mathcal{N})$为可测空间,$f:X\to Y$称为$(\mathcal{M},\mathcal{N})$-可测的,
    或可测的,若$\forall E\in\mathcal{N},f^{-1}(E)\in\mathcal{M}$
\end{definition}
可测函数的复合明显是可测的,$f:X\to Y$是$(\mathcal{M},\mathcal{N})$-可测的,\\
$g:Y\to Z$是$(\mathcal{N},\mathcal{O})$-可测的,则$E\in\mathcal{O}\Rightarrow g^{-1}(E)\in\mathcal{N}\Rightarrow f^{-1}(g^{-1}(E))\in\mathcal{M}$\\
即$(g\circ f)^{-1}(E)\in\mathcal{M}$,从而$g\circ f$是$(\mathcal{M},\mathcal{O})$-可测的
\begin{proposition}
    若$\mathcal{N}$是$\varepsilon$生成的$\sigma$-代数,则$f:X\to Y$是$(\mathcal{M},\mathcal{N})$-可测的充要条
    件是$\forall E\in\varepsilon,f^{-1}(E)\in\mathcal{M}$
    \begin{proof}
        由于$\varepsilon\subset\mathcal{N}$,必要性是明显的。充分性:$\left\{E\subset Y:f^{-1}(E)\in\mathcal{M}\right\}$是
        包含$\varepsilon$的$\sigma$-代数,从而包含$\mathcal{N}$
    \end{proof}
\end{proposition}
\begin{corollary}
    若$X,Y$是度量(拓扑)空间,所有连续映射$f:X\to Y$是$(\mathcal{B}_{X},\mathcal{B}_{Y})$
    -可测的
    \begin{proof}
        $f$是连续的$\Leftrightarrow\forall U\subset Y$为开集,$f^{-1}(U)\in\mathcal{B}_X$,而$\mathcal{B}_Y$由$Y$中的开集生
        成
    \end{proof}
\end{corollary}
设$(X,\mathcal{M})$是可测空间,若$X$上的实值或复值函数$f$是$(\mathcal{M},\mathcal{B}_{\mathbb{R}})$或$(\mathcal{M},\mathcal{B}_{\mathbb{C}})$
-可测的,则称$f$为$\mathcal{M}$-可测的或可测的。

若$f:\mathbb{R}\to\mathbb{C}$是$(\mathcal{L},\mathcal{B}_{\mathbb{C}})(\text{或}(\mathcal{B}_{\mathbb{R}},\mathcal{B}_{\mathbb{C}}))$-可测的,则称为Lebesgue(Borel)-可
测的,$f:\mathbb{R}\to\mathbb{R}$同理

$f,g:\mathbb{R}\to\mathbb{R}$是Lebesgue-可测的无法得出$f\circ g$是Lebesgue-可测的,因
为$E\in\mathcal{B}_{\mathbb{R}}\Rightarrow f^{-1}(E)\in\mathcal{L}$,却不一定$\in\mathcal{B}_{\mathbb{R}}$,若$f$是Borel-可测的,则$f\circ 
g$是Lebesgue-可测的
\begin{proposition}
    若$(X,\mathcal{M})$是可测空间,$f:X\to\mathbb{R}$,以下命题等价\\
    a. $f$是$\mathcal{M}$-可测的\\
    b. $f^{-1}((a,+\infty))\in\mathcal{M},\forall a\in\mathbb{R}$\\
    c. $f^{-1}([a,+\infty))\in\mathcal{M},\forall a\in\mathbb{R}$\\
    d. $f^{-1}((-\infty,a))\in\mathcal{M},\forall a\in\mathbb{R}$\\
    e. $f^{-1}((-\infty,a])\in\mathcal{M},\forall a\in\mathbb{R}$\\
    \begin{proof}
        由命题2.1.12. 命题3.1.2 可得
    \end{proof}
\end{proposition}
\begin{definition}
    设$(X,\mathcal{M})$是可测空间,$f:X\to\mathbb{R}$,$E\in\mathcal{M}$,我们称$f$是$E$上
    可测的,如果$\forall B\in\mathcal{B}_{\mathbb{R}},E\cap f^{-1}(B)\in\mathcal{M}$\\
    $\Leftrightarrow \forall B\in\mathcal{B}_{\mathbb{R}},f_{|E}^{-1}(B)\in\mathcal{M}_E:=\left\{F\cap E:F\in\mathcal{M}\right\}$
\end{definition}
给定集合$X$,若$\left\{(Y_{\alpha},\mathcal{N}_{\alpha})\right\}_{\alpha\in A}$是一族可测空间,$f_{\alpha}:X\to Y_{\alpha},\alpha\in 
A$则$X$上存在唯一一个最小的$\sigma$-代数,使得每个$f_{\alpha}$是可测的,这个$\sigma$-代数
由$f_{\alpha}^{-1}(E_{\alpha})$生成,其中$E_{\alpha}\in\mathcal{N}_{\alpha},\alpha\in A$,称为由$\left\{f_{\alpha}\right\}_{\alpha\in A}$生成的$\sigma$-代数,特
别的,$X=\prod_{\alpha\in A}Y_{\alpha}$时,$X$上的乘积$\sigma$-代数是投影映射$\left\{\pi_{\alpha}\right\}_{\alpha\in A}$生成的
\begin{proposition}
    设$(X,\mathcal{M})$和$(Y_{\alpha},\mathcal{N}_{\alpha}),\alpha\in A$是可测空间,$Y=\prod_{\alpha\in A}Y_{\alpha},\\
    \mathcal{N}=\bigotimes_{\alpha\in A}\mathcal{N}_{\alpha}$,$\pi_{\alpha}:Y\to Y_{\alpha}$是投影映射,那么$f:X\to Y$是可测的当且仅
    当$f_{\alpha}=\pi_{\alpha}\circ f$是$(\mathcal{M},\mathcal{N}_{\alpha})$-可测的
    \begin{proof}
        从上面的讨论知投影映射$\pi_{\alpha}:Y\to Y_{\alpha}$是$(\bigotimes_{\alpha\in A}\mathcal{N}_{\alpha},\mathcal{N}_{\alpha})$-可测的,
        若$f$是$(\mathcal{M},\mathcal{N})$可测的,那么$f_{\alpha}=\pi_{\alpha}\circ f$是$(\mathcal{M},\mathcal{N}_{\alpha})$-可测的,必要性成立\\
        若$f_{\alpha}$是$(\mathcal{M},\mathcal{N}_{\alpha})$-可测的,$\forall E_{\alpha}\in\mathcal{N}_{\alpha},f^{-1}_{\alpha}(E_{\alpha})=f^{-1}(\pi_{\alpha}^{-1}(E_{\alpha}))\in\mathcal{M}$\\
        由于$\mathcal{N}=\bigotimes_{\alpha\in A}\mathcal{N}_{\alpha}=\mathcal{M}(\left\{\pi_{\alpha}^{-1}(E_{\alpha}):E_{\alpha}\in\mathcal{N}_{\alpha},\alpha\in A\right\})$\\
        由命题3.1.2知$f$是$(\mathcal{M},\mathcal{N})$-可测的
    \end{proof}
\end{proposition}
\begin{corollary}
    $f:X\to\mathbb{C}$是$\mathcal{M}$-可测的当且仅当$Re(f),Im(f)$是$\mathcal{M}$-可测的
\end{corollary}
扩展实数系$\overline{\mathbb{R}}=\mathbb{R}\cup\left\{-\infty,+\infty\right\}$,定义$\overline{\mathbb{R}}$上的Borel集$\mathcal{B}_{\overline{\mathbb{R}}}=\left\{E\subset\overline{\mathbb{R}}:E\cap\mathbb{R}\in\mathcal{B}_{\mathbb{R}}\right\}$
$\mathcal{B}_{\overline{\mathbb{R}}}$可由$(a,+\infty],[-\infty,a)$生成。$f:X\to\overline{\mathbb{R}}$如果是$(\mathcal{M},\mathcal{B}_{\overline{\mathbb{R}}})$-可测的,则称为
是$\mathcal{M}$-可测的
\begin{proposition}
    若$f,g:X\to\mathbb{C}$是$\mathcal{M}$-可测的,那么$f+g,fg$也是$\mathcal{M}$-可测的
    \begin{proof}
        定义$F:X\to \mathbb{C}\times\mathbb{C},\ \phi,\psi:\mathbb{C}\times\mathbb{C}\to \mathbb{C}$\\
        $F(x)=(f(x),g(x)),\phi(z,w)=z+w,\psi(z,w)=zw$\\
        $\mathcal{B}_{\mathbb{C}\times\mathbb{C}}=\mathcal{B}_{\mathbb{C}}\otimes\mathcal{B}_{\mathbb{C}}$,由命题3.1.6知$F$是$(\mathcal{M},\mathcal{B}_{\mathbb{C}\times\mathbb{C}})$-可测的,$\phi,\psi$均为连续的,
        从而可测,$f+g=\phi\circ F,fg=\psi\circ F$也是可测的
    \end{proof}
\end{proposition}
\begin{proposition}
    若$\left\{f_j\right\}_{j\in\mathbb{N}}$是一列$(X,\mathcal{M})$上的扩展实值可测函数,那么\\
    $g_1(x)=\underset{j}{\sup}f_j(x),\ \ \ \ \ \ \ \ \ g_3(x)=\underset{j\to\infty}{\limsup}f_j(x)$\\
    $g_2(x)=\underset{j}{\inf}f_j(x),\ \ \ \ \ \ \ \ \ \ g_4(x)=\underset{j\to\infty}{\liminf}f_j(x)$\\
    均为可测的
    \begin{proof}
        首先验证$g_1^{-1}((a,+\infty])=\bigcup_1^{\infty}f_j^{-1}((a,+\infty])$:\\
        $x\in g^{-1}_1((a,+\infty])\Leftrightarrow a<g_1(x)\leq +\infty\\
        \Leftrightarrow \exists j: a<f_j(x)\leq +\infty$\\
        否则$\forall j:f_j(x)\leq a\Rightarrow g_1(x)=\underset{j}{\sup}f_j(x)\leq a$,矛盾\\
        于是$\Leftrightarrow x\in\bigcup_1^{\infty}f_j^{-1}((a,+\infty])$\\
        同理$g_2^{-1}([-\infty,a))=\bigcup_1^{\infty}f_j^{-1}([-\infty,a))$\\
        于是$g_1,g_2$可测\\
        $g_3(x)=\underset{j\to\infty}{\limsup}f_j(x)=\underset{k}{\inf}(\underset{j>k}{\sup}f_j(x))$是可测函数,$g_4$同理
    \end{proof}
\end{proposition}
\begin{corollary}
    若$f,g:X\to \overline{\mathbb{R}}$是可测的,那么$max(f,g),min(f,g)$也是可测的
\end{corollary}
\begin{corollary}
    若$\left\{f_j\right\}_{j\in\mathbb{N}}$是一列复值可测函数,且$f(x)=\lim_{j\to\infty}f_j(x)$存在,
    则$f$也是可测函数
\end{corollary}
\begin{definition}
    设$f:X\to\overline{\mathbb{R}}$,我们定义$f$的正部和负部:\\
    $f^{+}(x)=max(f(x),0),\ \ f^{-}(x)=max(-f(x),0)$\\
    则$f=f^{+}-f^{-}$若$f$是可测的,那么$f^{+},f^{-}$也是可测的\\
    若$f:X\to\mathbb{C}$ 我们有极分解:\\
    $f=(sgn f)|f|$,其中$sgn(z)=
    \begin{cases}
        \frac{z}{|z|}& \text{ if } z\neq 0 \\
        0& \text{ if } z=0
      \end{cases}$\\
      若$f$是可测的,那么$z\mapsto|z|$是连续的,从而$|f|$可测,$z\mapsto sgn(z)$在除原点
      外连续,$U\subset \mathbb{C}$为开集,则$sgn^{-1}(U)=V\cup\left\{0\right\}$或开集,于是$sgn(z)$是可测
      的,即$sgn(f)=sgn\circ f$是可测的
\end{definition}
\begin{definition}
    设$(X,\mathcal{M})$为可测空间,$E\subset X$,我们定义$E$的特征函数:\\
    $\mathcal{X}_E(x)=
    \begin{cases}
        1& \text{ if } x\in E \\
        0& \text{ if } x\notin E
      \end{cases}$\\
      容易验证$\mathcal{X}_E$可测当且仅当$E\in\mathcal{M}$
\end{definition}
\begin{definition}
    $X$上的简单函数是一些可测特征函数在复系数下的线性组合,
    我们不允许简单函数取值无穷。等价的有:$f:X\to\mathbb{C}$是简单函数当且仅当
    $f$是可测的,且$f$的像集为$\mathbb{C}$的有限点集,此时有:\\
    $f=\sum_1^nz_j\mathcal{X}_{E_j}$,其中$E_j=f^{-1}(\left\{z_j\right\})$,$range(f)=\left\{z_1,\dots,z_n\right\}$\\
    称为$f$的标准表示,其中各$E_j$是不交的,$\bigcup_1^nE_j=X$(某个$z_j=0$)
\end{definition}
若$f,g$为简单函数,则$f+g,fg$也是,因为$f+g,fg$仍为可测函数且像集为有限点集

下面讨论可测函数由简单函数逼近
\begin{theorem}
    设$(X,\mathcal{M})$为可测空间\\
    a. 若$f:X\to[0,+\infty]$是可测的,存在一列简单函数$\left\{\phi_n\right\}_{n\in\mathbb{N}}$满足:\\
    $0\leq\phi_1\leq\phi_2\leq\dots\leq f,\phi_n$逐点收敛于$f$,且在任意$f$有界的子集上,是一致
    收敛的\\
    b. 若$f:X\to\mathbb{C}$是可测的,存在一列简单函数$\left\{\phi_n\right\}_{n\in\mathbb{N}}$满足:\\
    $0\leq|\phi_1|\leq|\phi_2|\leq\dots\leq |f|,\phi_n$逐点收敛于$f$,且在任意$f$有界的子集上,是一致
    收敛的
    \begin{proof}
        a. 
        对于$n=0,1,2,\dots,0\leq k\leq 2^{2n}-1$\\
        令$E_n^k=f^{-1}((k2^{-n},(k+1)2^{-n}]),F_n=f^{-1}((2^n,+\infty])$\\
        令$\phi_n=\sum_{k=0}^{2^{2n}-1}k2^{-n}\mathcal{X}_{E_n^k}+2^n\mathcal{X}_{F_n}$\\
        首先验证单调性:\\
        $x\in[0,+\infty]$\\
        1. $f(x)=0,\phi_n(x)=\phi_{n+1}(x)=0$\\
        2. $x\in E_n^k,\phi_n(x)=k2^{-n},\\
        f(x)\in(k2^{-n},(k+1)2^{-n}]=(2k2^{-(n+1)},2(k+1)2^{-(n+1)}]$\\
        $\exists l=2k,2k+1,\dots,2(k+1),x\in E_{n+1}^k,\\
        \phi_{n+1}(x)=l2^{-(n+1)}\geq 2k2^{-(n+1)}=\phi_n(x)$\\
        3. $x\in F_n,f(x)\in(2^n,+\infty]=(2^n,2*2^n]\cup(2^{n+1},+\infty]\\
        =(2^{2n+1}*2^{-(n+1)},2^{2n+2}*2^{-(n+1)}]\cup(2^{n+1},+\infty]$\\
        于是$\phi_{n+1}(x)=2^{n+1}$或$l2^{-(n+1)}\geq 2^{2n+1}*2^{-(n+1)}=2^n=\phi_n(x)$\\
        总之有$\phi_{n+1}\geq \phi_n$\\
        现在验证收敛性:\\
        设$f$在$E$上有上界$2^N,\forall n>N,\forall x\in E,\\
        \phi(x)=k2^{-n}<f(x)\leq (k+1)2^{-n}=\phi_n(x)+2^{-n}$\\
        即$0<f(x)-\phi_n(x)\leq 2^{-n},\forall x\in E$\\
        于是$\phi_n$在$E$上一致收敛于$f$\\
        $\forall x\in X$,若$f(x)<+\infty$,$\phi_n$在$x$点的收敛以得证,若$f(x)=+\infty$\\
        $\forall n,x\in F_n,\phi_n(x)=2^n\to +\infty=f(x)$,于是$\phi_n$于$x$收敛\\
        b. 设$f=g+ih$,对$g^+,g^-,h^+,h^-$使用a.,\\
        得到$\psi_n^+,\psi_n^-,\zeta_n^+,\zeta_n^-$\\
        令$\phi_n=\psi_n^+-\psi_n^-+i(\zeta_n^+-\zeta_n^-)$\\
        对于$x\in X$,$g^+,g^-,h^+,h^-$分别至少有一个为0,\\
        不妨设$g^-(x)=h^-(x)=0$,于是$\psi_n^-(x)=\zeta_n^-(x)=0,$\\
        $|f(x)|=\sqrt{g^+(x)^2+h^+(x)^2}\geq\sqrt{\psi_{n+1}^+(x)^2+\zeta_{n+1}^+(x)^2}\\
        \geq\sqrt{\psi_{n}^+(x)^2+\zeta_{n}^+(x)^2}$\\
        即$|f(x)|\geq|\phi_{n+1}(x)|\geq|\phi_n(x)|$\\
        收敛性的部分是明显的
    \end{proof}
\end{theorem}
\begin{proposition}
    设$(X,\mathcal{M},\mu)$是测度空间,下面推断是有效的,当且仅当
    测度$\mu$是完全的\\
    a. $f$是可测的且$f=g$是$\mu$-几乎处处成立$\Rightarrow g$是可测的\\
    b. $f_n$是可测的,且$f_n\to f$是$\mu$-几乎处处成立的$\Rightarrow f$是可测的
    \begin{proof}
        首先设$\mu$是完全的\\
        a. 令$h=g-f$,则$\exists E\in \mathcal{M},\mu(E)=0:\forall x\notin E,h(x)=0$\\
        往证$g=f+h$可测,只需证$h$可测,只需证$\forall a\in\mathbb{R},h^{-1}((a,+\infty])\in\mathcal{M}$\\
        1. $a\geq 0,h(x)>a\geq 0\Rightarrow x\in E$,即$h^{-1}((a,+\infty])\subset E$\\
        由于$\mu$是完全的,知$h^{-1}((a,+\infty])\in\mathcal{M}$\\
        2. $a<0,h^{-1}((a,+\infty])=h^{-1}((a,0))\cup h^{-1}(\left\{0\right\})\cup h^{-1}((0,+\infty])\\
        =F_1\cup E^c\cup F_2\in\mathcal{M}$,其中$F_1,F_2$为$E$的子集$\in\mathcal{M}$,$E^c\in\mathcal{M}$\\
        b. 令$h=\underset{n\to\infty}{\limsup}f_n,h=f$是$\mu$-几乎处处成立的,由于$h$是可测的,由a. 
        知$f$是可测的\\
        若$\mu$不是完全的,对于$\mu$-0测集$E$,$\exists F\subset E,F\notin\mathcal{M}$\\
        $\mathcal{X}_X=\mathcal{X}_F$是$\mu$-几乎处处成立的,但$\mathcal{X}_X$可测,$\mathcal{X}_F$不可测,即推断a.不成立
    \end{proof}
\end{proposition}
\begin{proposition}
    设$(X,\mathcal{M},\mu)$是一测度空间,$(X,\overline{\mathcal{M}},\overline{\mu})$是其完备化若$f$是$X$上
    的$\overline{\mathcal{M}}$-可测函数,则存在$\mathcal{M}$-可测函数$g$满足$f=g$是$\overline{\mu}$-几乎处处成立的
    \begin{proof}
        设$E\in\overline{\mathcal{M}},\exists F\in\mathcal{M},N'$为$\mu$-零测集$N$的子集,$E=F\cup N'$\\
        $\mathcal{X}_E=\mathcal{X}_F$在$N'\setminus F$外成立,设$\left\{\phi_n\right\}$为一列$\overline{\mathcal{M}}$-可测简单函数逐点收敛于$f$\\
        对$\phi_n$执行上述操作,可得$\psi_n$为$\mathcal{M}$-可测简单函数,且在$E_n$外$\phi_n=\psi_n$处处成立
        ,其中$E_n$为$\mu$-零测集$N_n$的子集,令$N=\bigcup_n N_n$为$\mu$-零测集\\
        令$g=\underset{n\to \infty}{\limsup}\psi_n,\forall x\in N^c:f(x)=\underset{n\to\infty}{\lim}\phi_n(x)=\underset{n\to\infty}{\lim}\psi_n(x)=g(x)$\\
        即$f=g$是$\mu$-几乎处处成立的,且$g$是$\mathcal{M}$-可测的
    \end{proof}
\end{proposition}
\end{document}
