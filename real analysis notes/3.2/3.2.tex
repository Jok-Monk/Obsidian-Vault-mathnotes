\documentclass[12pt, a4paper, oneside]{ctexbook}
\usepackage{amsmath, amsthm, amssymb, bm, graphicx, hyperref, mathrsfs}

\title{{\Huge{\textbf{尸变函数抄书笔记}}}\\———主要参考Folland}
\author{Joker}
\date{\today}
\linespread{1.5}
\newtheorem{theorem}{定理}[section]
\newtheorem{definition}[theorem]{定义}
\newtheorem{lemma}[theorem]{引理}
\newtheorem{corollary}[theorem]{推论}
\newtheorem{example}[theorem]{例}
\newtheorem{proposition}[theorem]{命题}

\begin{document}
在这一节中,取定测度空间$(X,\mathcal{M},\mu)$,定义$L^+$为所有$X\to[0,+\infty]$的
可测函数空间
\begin{definition}
    设$\phi=\sum_1^na_j\mathcal{X}_{E_j}\in L^+$为简单函数,我们定义$\phi$的积分\\
    $\int\phi\,d\mu=\underset{1}{\overset{n}{\sum}}a_j\mu(E_j)$\\
    约定$0*\infty=0$\\
    设$A\in\mathcal{M}$,$\phi*\mathcal{X}_A=\underset{1}{\overset{n}{\sum}}a_j\mathcal{X}_{E_j}\mathcal{X_A}=\underset{1}{\overset{n}{\sum}}a_j\mathcal{X}_{E_j\cap A}$仍为简单函数\\
    $\int_{A}\phi\,d\mu:=\int\phi*\mathcal{X}_A\,d\mu$\\
    在这个约定下$\int\phi\,d\mu=\int_X\phi\,d\mu$\\
    有时$\int_A\phi\,d\mu$也写作$\int_A\phi(x)\,d\mu(x)$或$\int_A\phi$
\end{definition}
\begin{proposition}
    设$\phi,\psi\in L^+$为简单函数\\
    a. $c\geq 0\Rightarrow \int c\phi=c\int\phi$\\
    b. $\int(\phi+\psi)=\int\phi+\int\psi$\\
    c. $\phi\leq\psi\Rightarrow \int\phi\leq\int\psi$\\
    d. $A\mapsto \int_A\phi\,d\mu$是$\mathcal{M}$上的一个测度
    \begin{proof}
        设$\phi=\sum_1^na_j\mathcal{X}_{E_j},\psi=\sum_1^mb_j\mathcal{X}_{F_j}$\\
        a. $\int c\phi=\sum_1^n ca_j\mu(E_j)=c\sum_1^na_j\mu(E_j)=c\int\phi$\\
        b. 由于$X=\bigcup_1^n E_j=\bigcup_1^m F_j$为不交并,$E_j=\bigcup_{k=1}^m(E_j\cap F_k),F_j=\bigcup_{k=1}^n(E_k\cap 
        F_j)$也是不交并\\
        $\int\phi+\int\psi=\sum_{j=1}^na_j\mu(E_j)+\sum_{k=1}^mb_k\mu(F_k)\\
        =\sum_{j=1}^na_j\mu(\bigcup_{k=1}^m(E_j\cap F_k))+\sum_{k=1}^mb_k\mu(\bigcup_{j=1}^n(E_j\cap F_k))\\
        =\sum_{j,k}(a_j+b_k)\mu(E_j\cap F_k)\\
        =\int(\phi+\psi)$\\
        c. $E_j\cap F_k\neq\varnothing\Rightarrow a_j\leq b_k$\\
        $\int\phi=\sum_{j=1}^na_j\mu(E_j)=\sum_{j,k}a_j\mu(E_j\cap F_k)\leq\sum_{j,k}b_j\mu(E_j\cap F_k)=\int\psi$\\
        d. $\int_{\varnothing}\phi\,d\mu=\int\phi*\mathcal{X}_{\varnothing}\,d\mu=\sum_1^na_j\mu(\varnothing\cap E_j)=0$\\
        设$\left\{A_k\right\}\subset\mathcal{M}$为一列不交集,$A=\bigcup_1^{\infty} A_k\\
        \int_A\phi\,d\mu=\sum_{j=1}^na_j\mu(A\cap E_j)=\sum_{j=1}^na_j\mu(\bigcup_{k=1}^{\infty}(A_k\cap E_j))\\
        =\sum_{j=1}^na_j\sum_{k=1}^{\infty}\mu(A_k\cap E_j)\\
        =\sum_{k=1}^{\infty}\sum_{j=1}^na_j\mu(A_k\cap E_j)\\
        =\sum_{k=1}^{\infty}\int_{A_k}\phi\,d\mu$\\
        于是该映射是$\mathcal{M}$上的一个测度
    \end{proof}
\end{proposition}
现在我们将积分的定义扩张到$L^+$上
\begin{definition}
    若$f\in L^+$,$\int f\,d\mu:=\sup\left\{\int\phi\,d\mu:0\leq\phi\leq f,\phi\text{是简单函数}\right\}$\\
    由命题3.2.2.c.知,$f$为简单函数时,两个定义是等价的\\
    容易验证$f\leq g\Rightarrow \int f\leq\int g,\int cf=c\int f,c\in[0,+\infty]$
\end{definition}
\begin{theorem}{The Monotone Convergence Theorem}
    \\若$\left\{f_n\right\}\subset L^+,f_j\leq f_{j+1},f=\underset{n\to\infty}{\lim}f_n(=\sup_n f_n)$\\
    那么$\int f=\underset{n\to\infty}{\lim}\int f_n$
    \begin{proof}
        首先$\int f_n$是单调递增数列,故极限存在或$=+\infty$\\
        $f_n\leq f\Rightarrow \int f_n\leq\int f\Rightarrow \underset{n\to\infty}{\lim}\int f_n\leq \int f$\\
        取$\alpha\in(0,1)$,$0\leq \phi\leq f$为简单函数,$E_n:=\left\{x:f_n(x)\geq\alpha\phi(x)\right\}$\\
        则$E_n\in\mathcal{M}(f-\alpha\phi$可测$(f-\alpha\phi)^{-1}[0,+\infty]\in\mathcal{M})$,且$E_n\subset E_{n+1},\bigcup_nE_n=X$,
        否则$\exists x\in X:\forall n,f_n(x)<\alpha\phi(x)\Rightarrow f(x)\leq\alpha\phi(x)<\phi(x)\leq f(x)$矛盾
        于是$\int_X f_n\geq\int_{E_n}f_n\geq\alpha\int_{E_n}\phi$\\
        $\underset{n\to\infty}{\lim}\int f_n\geq\alpha\underset{n\to\infty}{\lim}\int_{E_n}\phi=\alpha\int_{\bigcup_1^{\infty}E_n}\phi=\alpha\int\phi$\\
        以上讨论对任意$\alpha,\phi$成立,令$\alpha\to 1$,得$\underset{n\to\infty}{\lim}\int f_n\geq\int\phi$\\
        再对所有简单函数$\phi$取上确界,得$\underset{n\to\infty}{\lim}\int f_n\geq\int f$\\
        综上,$\underset{n\to\infty}{\lim}\int f_n=\int\phi$
    \end{proof}
\end{theorem}
对于$f\in L^+$,可取$\left\{\phi\right\}\subset L^+$为一单调简单函数且逐点收敛于$f$,$\int f=\lim_{n\to\infty}\int \phi_n$
\begin{theorem}
    若$\left\{f_n\right\}\subset L^+$是一列有限或可数函数,$f=\sum_nf_n$\\
    则$\int f=\sum_n\int f_n$
    \begin{proof}
        首先考虑$f_1,f_2$,取$\left\{\phi_j\right\},\left\{\psi_j\right\}\subset L^+$为单调递增简单函数且分别逐点
        收敛于$f_1,f_2$,那么$\left\{\phi_j+\psi_j\right\}$为单调递增简单函数且逐点收敛于$f_1+f_2$\\
        $\int(f_1+f_2)=\lim_{j\to\infty}\int(\phi_j+\psi_j)=\lim_{j\to\infty}(\int\phi_j+\int\psi_j)=\lim_{j\to\infty}\int\phi_j+
        \lim_{j\to\infty}\int\psi_j=\int f_1+\int f_2$\\
        这表明若$\left\{f_n\right\}$是有限的,$\int f=\sum_n \int f_n$\\
        若是无穷的,$\sum_1^Nf_n$是单调递增的,$\int f=\int\sum_1^{\infty}f_n=\int\underset{N\to\infty}{\lim}\sum_1^Nf_n\\
        =\lim_{N\to\infty}\int\sum_1^Nf_n=\lim_{N\to\infty}\sum_1^N\int f_n=\sum_1^{\infty}\int f_n$
    \end{proof}
\end{theorem}
\begin{proposition}
    若$f\in L^+,$那么$\int f=0\Leftrightarrow f=0$ a.e.
    \begin{proof}
        首先若$f=\sum_ja_j\mathcal{X}_{E_j}$是简单函数,$\int f=0\Leftrightarrow a_j=0$或$\mu(E_j)=0$\\
        对每个$j$成立,$\Leftrightarrow f=0$ a.e.\\
        $\Leftarrow$\\
        $0\leq\phi\leq f$为简单函数,则$\phi=0$ a.e.$\Rightarrow\int \phi=0$\\
        于是$\int f=\sup_{0\leq\phi\leq f} \int \phi=0$\\
        $\Rightarrow$\\
        $f^{-1}((0,+\infty])=f^{-1}((\bigcup_1^{\infty}(\frac{1}{n}),+\infty])=\bigcup_1^{\infty}f^{-1}((\frac{1}{n},+\infty])$\\
        若$f=0$ a.e.不成立,必$\exists N:\mu(f^{-1}((\frac{1}{N},+\infty]))>0$,记为$E_N$\\
        否则$\mu(f^{-1}((0,+\infty]))\leq\sum_1^{\infty}\mu(f^{-1}((\frac{1}{n},+\infty]))=0$\\
        这与$f=0$ a.e.不成立矛盾\\
        于是$\int f\geq\int f*\mathcal{X}_{E_N}>\int \frac{1}{N}*\mathcal{X}_{E_N}=\frac{1}{N}*\mu(E_N)>0$\\
        这又与$\int f=0$矛盾
    \end{proof}
\end{proposition}
\begin{corollary}
    若$\left\{f_n\right\}\subset L^+,f\in L^+$,$f_n(x)$单调递增趋于$f(x)$ a.e.那么$\int f=\lim_{n\to\infty}\int f_n$
    \begin{proof}
        $\exists E\in\mathcal{M}:\mu(E^c)=0,\forall x\in E,f_n(x)$单调递增趋于$f(x)$成立,那
        么$f-f*\mathcal{X}_E\in L^+,f_n-f_n*\mathcal{X}_E\in L^*$且$f-f*\mathcal{X}_E=0,f_n-f_n*\mathcal{X}_E=0$ a.e.,
        由命题3.2.6.知$\int f-f*\mathcal{X}_E=\int f_n-f_n*\mathcal{X}_E=0$\\
        又$f_n*\mathcal{X}_E$单调递增趋于$f*\mathcal{X}_E$\\
        $\int f=\int f*\mathcal{X}_E=\lim_{n\to\infty}\int f_n*\mathcal{X}_E=\lim_{n\to\infty}\int f_n$
    \end{proof}
\end{corollary}
\begin{lemma}{Fatou's Lemma}
    \\若$\left\{f_n\right\}\subset L^+$是任意序列,则$\int(\liminf f_n)\leq \liminf \int f_n$
    \begin{proof}
        $\forall j\geq k,\underset{n\geq k}{\inf}f_n\leq f_j,$于是$\int \underset{n\geq k}{\inf}f_n\leq \int f_j,\forall j\geq k$\\
        $\Rightarrow \int \underset{n\geq k}{\inf}f_n\leq\underset{j\geq k}{\inf}\int f_j$\\
        令$k\to \infty$,有$\int (\liminf f_n)=\lim_{k\to\infty}\int \underset{n\geq k}{\inf}f_n\leq\liminf \int f_n$
    \end{proof}
\end{lemma}
\begin{corollary}
    若$\left\{f_n\right\}\subset L^+,f\in L^+,f_n\to f$ a.e.,那么$\int f\leq \liminf\int f_n$
    \begin{proof}
        若$f_n\to f$处处成立,那么根据Fatou's Lemma,$\int f=\int \liminf f_n\leq
        \liminf \int f_n$。一般的,$\exists E\in\mathcal{M}:\mu(E^c)=0,\forall x\in E,f_n(x)\to f(x)$,即$f_n*
        \mathcal{X}_E\to f*\mathcal{X}_E$处处成立,$f-f*\mathcal{X}_E,f_n-f_n*\mathcal{X}_E\in L^+,f-f*\mathcal{X}_E=0,f_n-
        f_n*\mathcal{X}_E=0$ a.e.,于是$\int f=\int f*\mathcal{X}_E,\int f_n=\int f*\mathcal{X}_E\\
        \int f=\int f*\mathcal{X}_E\leq \liminf \int f_n*\mathcal{X}_E=\liminf \int f_n$
    \end{proof}
\end{corollary}
\begin{proposition}
    若$f\in L^+,\int f<+\infty$,那么$\left\{x:f(x)=+\infty\right\}$是零测集,且
    $\left\{x:f(x)>0\right\}$是$\sigma$-有限的
    \begin{proof}
        记$E_n=f^{-1}((n,n+1]),n=1,\dots,F_k=f^{-1}((\frac{1}{k+1},\frac{1}{k}]),k=1,2,\dots
        $,则$\left\{x:f(x)>0\right\}=\bigcup_{n=1}^{\infty}E_n\cup\bigcup_{k=1}^{\infty}F_k$是不交并,若$\exists E_n$或$F_k$,使得$\mu(E_n)=
        +\infty$或$\mu(F_k)=+\infty$,则取$\phi=n\mathcal{X}_{E_n}$或$\frac{1}{k+1}\mathcal{X}_{F_k}$,均有$\phi\leq f$,于是$\int f\geq
        \int \phi=+\infty$,矛盾,从而$\left\{x:f(x)>0\right\}$是$\sigma$-有限的\\
        记$E=f^{-1}(\left\{+\infty\right\})$,若$\mu(E)=c>0$,那么取$\phi_n=n\mathcal{X}_E$,有$\phi_n\leq f\\
        \int f\geq\lim_{n\to\infty}\int \phi_n=\lim_{n\to\infty}nc=+\infty$,矛盾,故$\mu(E)=0$
    \end{proof}
\end{proposition}
\end{document}
