\documentclass[12pt, a4paper, oneside]{ctexbook}
\usepackage{amsmath, amsthm, amssymb, bm, graphicx, hyperref, mathrsfs}

\title{{\Huge{\textbf{尸变函数抄书笔记}}}\\———主要参考Folland}
\author{Joker}
\date{\today}
\linespread{1.5}
\newtheorem{theorem}{定理}[section]
\newtheorem{definition}[theorem]{定义}
\newtheorem{lemma}[theorem]{引理}
\newtheorem{corollary}[theorem]{推论}
\newtheorem{example}[theorem]{例}
\newtheorem{proposition}[theorem]{命题}

\begin{document}
这节主要构建了$\mathbb{R}$上的Borel测度,即定义域为$\mathcal{B}_{\mathbb{R}}$的测度\\
设$\mu$为$\mathbb{R}$上的有限Borel测度,令$F(x)=\mu((-\infty,x])$,则$F(x)$是$\mathbb{R}$上的递增右连续函数
$((-\infty,x]=\bigcap_1^{\infty}(-\infty,x_n],x_n\text{严格单调下降趋于}x,F(x)=\mu((-\infty,x])
=\mu(\bigcap_1^{\infty}(-\infty,x_n])=\lim_{n\to\infty}\mu((-\infty,x_n])=\lim_{n\to\infty}F(x_n),\text{由海涅定理知}F\text{右连续})$\\
于是我们可以通过$\mathbb{R}$上的递增右连续函数来构造Borel测度\\
以下形如$(a,b]$的区间简称为h-区间\\
容易验证,所有h-区间构成elementry family,故所有h-区间的不交并构成代数$\mathcal{A}$,且其生成的$\sigma$-代数为$\mathcal{B}_{\mathbb{R}}$\\
\begin{proposition}
    设$F:\mathbb{R}\to\mathbb{R}$为单调递增右连续函数,若$(a_j,b_j](j=1,\dots n)$为不交的h-区间
    令\\
    $\mu_0(\bigcup_1^n(a_j,b_j])=\sum_1^n[F(b_j)-F(a_j)],\mu_0(\phi)=0$\\
    则$\mu_0$为$\mathcal{A}$上的预测度
    \begin{proof}
        首先验证$\mu_0$是良定义的\\
        若$\left\{(a_j,b_j]\right\}_1^n$为不交的,且$\bigcup_1^n(a_j,b_j]=(a,b]$\\
        不妨设$a=a_1<b_1=a_2<\dots<b_n=b$,故$\sum_1^n[F(b_j)-F(a_j)]=F(b)-F(a)$
        更一般的,设$\left\{I_i\right\}_1^n,\left\{J_j\right\}_1^m$为有限不交h-区间,$\bigcup_1^nI_i=\bigcup_1^mJ_j$\\
        $\sum_i\mu_0(I_i)=\sum_{i,j}\mu_0(I_i\cap J_j)=\sum_j\mu_0(J_j)$\\
        故$\mu_0$是良定义的\\
        由于$F$单调递增,$F(+\infty),F(-\infty)$在$\mathbb{R}\cup\left\{+\infty,-\infty\right\}$上有定义\\
        由定义知$\mu_0$是有限可加的\\
        设$\left\{I_i\right\}_1^{\infty}\subset\mathcal{A}$是不交的,$\bigcup_1^{\infty}I_i=\bigcup_1^mJ_j\in\mathcal{A}$,其中$J_j$是不交的\\
        $\bigcup_1^{\infty}I_i=\bigcup_{j=1}^m(J_j\cap\bigcup_{i=1}^{\infty}I_i)=\bigcup_{j=1}^m\bigcup_{i=1}^{\infty}J_j\cap I_i$\\
        $\mu_0(\bigcup_1^{\infty}I_i)=\sum_{j=1}^m\mu_0(\bigcup_{i=1}^{\infty}J_j\cap I_i)$\\
        $\mu_0(\bigcup_1^mJ_j)=\sum_1^m\mu_0(J_j)$\\
        往证$\mu_0(\bigcup_1^{\infty}I_i)=\mu_0(\bigcup_1^mJ_j)$,只需证$\mu_0(\bigcup_{i=1}^{\infty}J_j\cap I_i)=\mu_0(J_j),j=1,\dots m$\\
        于是不妨设$\bigcup_1^{\infty}I_i=J$为h-区间\\
        $\mu_0(J)=\mu_0(\bigcup_1^nI_i)+\mu_0(J\setminus\bigcup_1^nI_i)\geq\mu_0(\bigcup_1^nI_i)=\sum_1^n\mu_0(I_i)$\\
        $n\to\infty,\mu_0(J)\geq\sum_1^n\mu_0(I_i)$\\
        为证明$\mu_0(J)\leq\sum_1^{\infty}\mu_0(I_i)$\\
        首先假设$J=(a,b],-\infty<a<b<+\infty$\\
        $\forall \epsilon>0$,由$F$右连续,$\exists \delta,\delta_i>0,F(a+\delta)-F(a)<\epsilon,\\
        F(b_i+\delta_i)-F(b_i)<\epsilon 2^{-i}$,$\left\{(a_i,b_i+\delta_i)\right\}_1^{\infty}$覆盖$[a+\delta,b]$,于是存在有限子覆盖
        $\left\{(a_j,b_j+\delta_j)\right\}_1^N$可设$b_j+\delta_j\in(a_{j+1},b_{j+1}+\delta_{j+1})?$且诸开区间无包含关系\\
        $\mu_0(J)=F(b)-F(a)<F(b)-F(a+\delta)+\epsilon$\\
        $\leq F(b_N+\delta_N)-F(a_1)+\epsilon$\\
        $=F(b_N+\delta_N)-F(a_N)+\sum_1^{N-1}[F(a_{j+1})-F(a_j)]+\epsilon$\\
        $<\sum_1^N[F(b_j)+\epsilon 2^{-j}-F(a_j)]+\epsilon$\\
        $<\sum_1^{\infty}[F(b_j)-F(a_j)]+2\epsilon$\\
        $=\sum_1^{\infty}\mu_0(I_j)+2\epsilon$
        由$\epsilon$的任意性知$\mu_0(J)\leq \sum_1^{\infty}\mu_0(I_i)$\\
        $a=-\infty$时,任意$M<+\infty$,对$(-M.b]$进行上述过程,有:\\
        $F(b)-F(-M)\leq\sum_1^{\infty}\mu_0(I_i)+\epsilon$\\
        $\epsilon\to 0,M\to +\infty$,$\mu_0(J)\leq\sum_1^{\infty}\mu_0(I_i)$\\
        $b=+\infty$时,任意$M<+\infty$,对$(a,M]$进行上述过程,\\
        同理可得$F(M)-F(a)\leq\sum_1^{\infty}\mu_0(I_i)+2\epsilon$\\
        $\epsilon\to 0,M\to +\infty$,即得$\mu_0(J)\leq \sum_1^{\infty}\mu_0(I_i)$

    \end{proof}
\end{proposition}
\begin{theorem}
    若$F:\mathbb{R}\to\mathbb{R}$是任意递增右连续函数,则存在唯一$\mathbb{R}$上的Borel测度$\mu_{F}$
    使得$\mu_{F}((a,b])=F(b)-F(a)$,若$G$是另一递增右连续函数,$\mu_F=
    \mu_G$当且仅当$F-G$为常数,若$\mu$是$\mathbb{R}$上的Borel测度,且于任意有界Boerl集上
    有限,定义$F(x)=
    \begin{cases}
      \mu((0,x])& \text{ if } x>0 \\
      0& \text{ if } x=0 \\
      -\mu((x,0])& \text{ if } x<0
    \end{cases}
    $,则$F$为递增右连续函数,
    且$\mu=\mu_F$
    \begin{proof}
        首先由命题2.4.1,$F$可诱导$\mathcal{A}$上的一个预测度,且$F$和$G$诱导同一个
        预测度当且仅当$F-G$为常数,且这些预测度是$\sigma$-有限的,$\mathbb{R}=\bigcup_1^{+\infty}(j,j+1]$
        由命题2.3.7知前两个断言的正确性。最后一个断言$F$明显是递增的,
        任取$x_n$递减趋于$x$,$x\geq0$时,$(0,x]=\bigcap_1^{\infty}(0,x_n]$,$x<0$时,$(x,0]=\bigcup_1^{\infty}(x_n,0]$
        由海涅定理可得$F$的右连续性。$\mu=\mu_F$在$\mathcal{A}$上成立,于是由命题2.3.7知其
        在$\mathcal{B}_{\mathbb{R}}$上成立
    \end{proof}
\end{theorem}
若$F$是$\mathbb{R}$上的递增右连续函数,则$F$可以诱导一个定义域包含$\mathcal{B}_{\mathbb{R}}$的完全
的测度,记作$\mu_F$,称作$F$诱导的Lebesgue-Stieltjes测度。之后$\mu_F$简记为$\mu$,
其定义域记为$\mathcal{M}_{\mu}$,$\forall E\in\mathcal{M}_{\mu}$\\
$\mu(E)=\inf\left\{\sum_1^{\infty}[F(b_j)-F(a_j)]:E\subset\bigcup_1^{\infty}(a_j,b_j]\right\}$\\
$=\inf\left\{\sum_1^{\infty}\mu((a_j,b_j]):E\subset\bigcup_1^{\infty}(a_j,b_j]\right\}$\\
\begin{lemma}
    $\forall E\in\mathcal{M}_{\mu}$,\\
    $\mu(E)=\inf\left\{\sum_1^{\infty}\mu((a_j,b_j)):E\in\bigcup_1^{\infty}(a_j,b_j)\right\}$
    \begin{proof}
        等式右边的式子记作$\nu(E)$,设$E\subset\bigcup_1^{\infty}(a_j,b_j),(a_j,b_j)=\bigcup_{k=1}^{\infty}I_j^k,$
        其中$I_j^k=(c_j^k,c_j^{k+1}],c_j^1=a_j,k\to\infty,c_j^k$严格单调递增趋于$b_j$,于是\\
        $\sum_1^{\infty}\mu((a_j,b_j))=\sum_{j,k=1}^{\infty}\mu(I_j^K)\geq\mu(E)$,从而$\nu(E)\geq\mu(E)$\\
        另一方面,$\forall\epsilon>0,\exists\left\{(a_j,b_j]\right\}_1^{\infty}:E\subset\bigcup_1^{\infty}(a_j,b_j],\sum_1^{\infty}\mu((a_j,b_j])\leq\mu(E)+
        \epsilon$,对每个$j$,$\exists \delta_j>0,F(b_j+\delta_j)-F(b_j)<\epsilon2^{-j},E\subset\bigcup_1^{\infty}(a_j,b_j+\delta_j)$\\
        $\sum_1^{\infty}\mu((a_j,b_j+\delta_j))\leq\sum_1^{\infty}\mu((a_j,b_j+\delta_j])\leq\sum_1^{\infty}\mu((a_j,b_j])+\epsilon\leq\mu(E)+2\epsilon$\\
        故$\nu(E)\leq\mu(E)$
    \end{proof}
\end{lemma}
\begin{theorem}
    若$E\in\mathcal{M}_{\mu}$,则\\
    $\mu(E)=\inf\left\{\mu(U):U\supset E,U\text{是开集}\right\}\\
    =\sup\left\{\mu(K):K\subset E,K\text{是紧集}\right\}$
    \begin{proof}
        由引理2.4.3,$\forall\epsilon>0,\exists(a_j,b_j):E\subset\bigcup_1^{\infty}(a_j,b_j),\sum_1^{\infty}\mu((a_j,b_j))<\mu(E)
        +\epsilon$,令$U=\bigcup_1^{\infty}(a_j,b_j)$,$U$是开集,$\mu(U)\leq\sum_1^{\infty}\mu((a_j,b_j))<\mu(E)+
        \epsilon$,另一方面$E\subset U\Rightarrow\mu(E)\leq\mu(U)$,故第一个等式成立\\
        现证第二个等式\\
        首先设$E$有界,若$E$是闭集,则也是紧集,等式自然成立,否则,$\forall\epsilon>0,
        \exists $开集$U:U\supset \overline{E}\setminus E,\mu(U)<\mu(\overline{E}\setminus E)+\epsilon$($\overline{E}$是闭集,故$\in\mathcal{B}_{\mathbb{R}}\subset\mathcal{M}_{\mu}$,从
        而$\overline{E}\setminus E\in\mathcal{M}_{\mu}$)令$K=\overline{E}\setminus U=E\setminus U$,则$K\subset E$为紧集,\\
        $\mu(K)=\mu(E)-\mu(E\cap U)=\mu(E)-[\mu(U)-\mu(U\setminus E)]\\
        =\mu(E)-\mu(U)+\mu(U\setminus E)\geq\mu(E)-\mu(\overline{E}\setminus E)+\mu(U\setminus E)-\epsilon\geq\mu(E)-\epsilon$
        若$E$无界,令$E_j=E\cap(j,j+1]$,对任意$\epsilon>0,\exists K_j\subset E_j$为紧集,$\mu(K_j)\geq
        \mu(E_j)-\epsilon2^{-|j|}$,令$H_n=\bigcup_{-n}^nK_j$,则$H_n\subset E$为紧集,$\mu(H_n)=\sum_{-n}^n\mu(K_j)\geq
        \sum_{-n}^n\mu(E_j)-3\epsilon\geq\mu(\bigcup_{-n}^nE_j)-3\epsilon$\\
        $\mu(E)=\lim_{n\to\infty}\mu(\bigcup_{-n}^nE_j),\exists N:\mu(\bigcup_{-n}^nE_j)>\mu(E)-\epsilon$对所有$n\geq N$成
        立,于是$\mu(H_N)\geq\mu(E)-4\epsilon$
    \end{proof}
\end{theorem}
\begin{theorem}
    若$E\subset\mathbb{R}$,那么以下命题等价:\\
    a. $E\in\mathcal{M}_{\mu}$\\
    b. $E=V\setminus N_1$,其中$V$是$G_{\delta}$集,$\mu(N_1)=0$\\
    c. $E=H\cup N_2$,其中$H$为$F_{\sigma}$集,$\mu(N_2)=0$
    \begin{proof}
        由于$\mu$在$\mathcal{M}_{\mu}$上是完全的,故b.c.$\Rightarrow$a.是明显的\\
        若$\mu(E)<+\infty$,\\
        由定理2.4.4,对$j\in\mathbb{N}$,$\exists$开集$U_j\supset E$,紧集$K_j\subset E$\\
        $\mu(U_j)-2^{-j}\leq\mu(E)\leq\mu(K_j)+2^{-j}$\\
        令$V=\bigcap_1^{\infty}U_j,H=\bigcup_1^{\infty}K_j$,则$H\subset E\subset V$\\
        $\mu(V)-2^{-j}\leq\mu(U_j)-2^{-j}\leq\mu(E)\leq\mu(K_j)+2^{-j}\leq\mu(H)+2^{-j}$\\
        $j\to\infty,\mu(V)=\mu(E)=\mu(H)<+\infty$\\
        故$\mu(V\setminus E)=\mu(E\setminus H)=0$\\
        若$\mu(E)=+\infty$,\\
        ?
    \end{proof}
\end{theorem}
\begin{proposition}
    若$E\in\mathcal{M}_{\mu},\mu(E)<+\infty$,则$\forall\epsilon>0,\exists A$为有限个开区间的并,
    且$\mu(E\triangle A)<\epsilon$
    \begin{proof}
        ?
    \end{proof}
\end{proposition}
\begin{definition}
    $F(x)=x$诱导的完全的测度$\mu_{F}$称为勒贝格测度,记作$m$,$m$定
    义域中的集合称为勒贝格可测集,记作$\mathcal{L}$
\end{definition}
\begin{definition}
    $E\subset\mathbb{R},r,s\in\mathbb{R}$\\
    $E+s:=\left\{x+s:x\in E\right\},rE:=\left\{rx:x\in E\right\}$
\end{definition}
\begin{theorem}
    若$E\in\mathcal{L}$,则$E+s\in\mathcal{L},rE\in\mathcal{L},\forall r,s\in\mathbb{R}$,且$m(E+s)=
    m(E),m(rE)=|r|m(E)$
    \begin{proof}
        首先变换$E\mapsto E+s,E\mapsto rE$是保持$\mathcal{A}$的双射($r=0$时明显保持Borel集,
        故不妨设$r\neq 0$),从而也是保持$\mathcal{B}_{\mathbb{R}}$的?,其中$\mathcal{A}$为h-区间有限不交并的集合,
        $E$为h-区间时,$m(E)= m(E+s),m(rE)=|r|m(E)$是明显的,又$m$是$\sigma$-有
        限的,由定理2.3.7知:\\
        $m(E+s)=m(E),m(rE)=|r|m(E)$在$\mathcal{B}_{\mathbb{R}}$上恒成立\\
        于是$E\in\mathcal{L},m(E)=0\Rightarrow \exists U_j\supset E$为开集$,m(U_j)<2^{-j},\\
        U_j+s\supset E+s,rU_j\supset rE,\\
        m(E+s)\leq m(U_j+s)<2^{-j}\\
        m(rE)\leq m(rU_j)<2^{-j}$\\
        从而$m(E+s)=m(rE)=0$\\
        $E\in\mathcal{L}$可写成Borel集和勒贝格零测集的并:$E=H\cup N$,
        于是$E+s=(H+s)\cup(N+s)\in\mathcal{L},rE=(rH)\cup(rN)\in\mathcal{L}$,且$m(E+s)=m(E),m(rE)=|r|m(E)$成立
    \end{proof}?
    \begin{proof}
        记$m^*$为$F(x)=x$诱导的$\mathbb{R}$上的外测度,$r=0$时是明显的,故不妨
        设$r\neq 0$,容易验证:\\
        $(A+s)\cap(B+s)=(A\cap B)+s\\
        (A+s)\cup(B+s)=(A\cup B)+s\\
        (A+s)^c=A^c+s\\
        (rA)\cap(rB)=r(A\cap B)\\
        (rA)\cup(rB)=r(A\cup B)\\
        (rA)^c=r(A^c)\\
        E\in\mathcal{A}\Rightarrow E+s,rE\in\mathcal{A}$\\
        首先验证,$\forall E\subset\mathbb{R},m^*(E+s)=m^*(E),m^*(rE)=|r|m^*(E):$\\
        对于$h$-区间$E$,$\mu_F(E+s)=\mu_F(E),\mu_F(rE)=|r|\mu_F(E)$是明显的\\
        $\forall E\subset\mathbb{R},m^*(E+s)=\inf\left\{\sum_1^{\infty}\mu_F(I_j):I_j\in\mathcal{A},E+s\subset\bigcup_1^{\infty}I_j\right\}\\
        > \sum_1^{\infty}\mu_F(I_j)-\epsilon=\sum_1^{\infty}\mu_F(I_j-s)-\epsilon$,其中$I_j-s\in\mathcal{A},E\subset\bigcup_1^{\infty}(I_j-s)$\\
        即$m^*(E+s)\geq m^*(E)-\epsilon$对$\forall\epsilon>0$成立,故$m^*(E+s)\geq m^*(E)$\\
        同理$m^*(E+s)\leq m^*(E)$\\
        由于$rE\subset\bigcup_1^{\infty}I_j\Rightarrow E\subset\bigcup_1^{\infty}(r^{-1}I_j),r^{-1}I_j\in\mathcal{A}$\\
        同上讨论知$m^*(rE)=|r|m^*(E)$\\
        设$E\in\mathcal{L}$\\
        $E+s\in\mathcal{L}\Leftrightarrow m^*(A)=m^*(A\cap(E+s))+m^*(A\cap(E+s)^c),\forall A\subset\mathbb{R}\\
        \Leftrightarrow m^*(A)=m^*((A-s)\cap E+s)+m^*((A-s)\cap(E^c)+s)\\
        \Leftrightarrow m^*(A)=m^*((A-s)\cap E)+m^*((A-s)\cap E^c)\\
        \Leftrightarrow m^*(A)=m^*(A-s)$成立\\
        同理可得$m^*(A)=m^*(A\cap(rE))+m^*(A\cap(rE)^c)$\\
        从而$E+s,rE\in\mathcal{L}$\\
        对于$E\in\mathcal{L}$\\
        由于$m(E)=\inf\left\{\sum_1^{\infty}\mu_F(a_j,b_j):E\subset\bigcup_1^{\infty}(a_j,b_j)\right\}$\\
        同上讨论可知$m(E+s)=m(E),m(rE)=|r|m(E)$
    \end{proof}
\end{theorem}
[0,1]上的p进位表数法\\
$x\in[0,1]$,我们按以下方式将$x$唯一的写做$\sum_1^{\infty}a_jp^{-j}$,其中$a_j=0,\dots,p-1$
$p$为大于1的整数\\
若$x=0,x=\sum_1^{\infty}0*p^{-j}$\\
若$x\in(0,1]$\\
区间$(kp^{-j},(k+1)p^{-j}]$记作$I_j^k$,其中$k=0,1,\dots,p-1,j=1,\dots$\\
第1步:\\
$I_0^0=\bigcup_0^{p-1}I_1^k$是不交并,于是$\exists !k:x\in I_1^k,a_1:=k$\\
第2步:\\
$a_1p^{-1}+I_1^{0}=\bigcup_0^{p-1}(a_1p^{-1}+I_2^k)$是不交并,于是$\exists !k,x\in a_1p^{-1}+I_2^k,a_2:=k$\\
$\dots$\\
第$j$步:\\
设已经取出$a_1,a_2,\dots a_{j-1}\\
x\in\sum_{i=1}^{j-1}a_ip^{-i}+I_{j-1}^{0}=\bigcup_{k=1}^{p-1}(\sum_{i=1}^{j-1}a_ip^{-i}+I_j^k)$为不交并,\\
于是$\exists !k:x\in \sum_{i=1}^{j-1}a_ip^{-i}+I_j^k,a_j:=k$\\
$\dots$\\
我们得到唯一的$\left\{a_j\right\}_1^{\infty}$满足:\\
$a_j=0,1,\dots,p-1\\
\sum_{i=1}^{j}a_1p^{-i}<x\leq\sum_{i=1}^{j}a_ip^{-i}+p^{-j},j=1,2,\dots\\
\forall j>0,\exists k\geq j,a_k\neq 0$\\
否则,设$\exists j>0,\forall k\geq j,a_k=0$\\
于是$\sum_{i=1}^{j-1}a_1p^{-i}<x\leq\sum_{i=1}^{j-1}a_ip^{-i}+p^{-k},k=j-1,j,j+1,\dots$\\
$k\to\infty$ 得$\sum_{i=1}^{j-1}a_1p^{-i}<x\leq\sum_{i=1}^{j-1}a_ip^{-i}$矛盾\\
于是$x=\sum_1^{\infty}a_ip^{-i}$,记作$0.a_1a_2\dots a_j\dots$为无穷小数\\
下面的讨论中,$x$的p进小数总采用以上取法\\
\begin{definition}
    Cantor 集$C:=\left\{x\in[0,1]:x\text{的3进小数}0.a_1a_2\dots\text{中}a_j\neq 1,\notag \right.\\ \left.
    \text{或}\exists N>0,a_N=1,a_j\neq 1(j<N),a_j=2(j>N),j=1,\dots\right\}$\\
    注意Cantor集等价于$[0,1]$去掉形如$(0.a_1\dots a_j1,\ 0.a_1\dots a_j2)$的开区间,其中$a_i\neq 1,i=1,\dots j$\\
\end{definition}
\begin{proposition}
    设C为Cantor集\\
    a. C是紧的,无处稠密的,完全不连通的(连通部分为单点集),没有孤立点的\\
    b. $m(C)=0$\\
    c. $card(C)=\mathcal{C}$
    \begin{proof}
        b. 记$F=\bigcup\left\{(0.a_1\dots a_j1,\ 0.a_1\dots a_j2):a_i\neq 1,i=1,\dots j\right\}\in\mathcal{B}_{\mathbb{R}}$\\
        则$[0,1]=C\cup F$是不交并,$C=[0,1]\cap F^c\in\mathcal{B}_{\mathbb{R}}$,$m([0,1])=m(C)+m(F)$\\
        $m(F)=\sum_0^{\infty}\frac{2^j}{3^{j+1}}=\frac{1}{3}*\frac{1}{1-\frac{2}{3}}=1$\\
        于是$m(C)=1-m(F)=0$\\
        c. 设$x\in C$,$x=0.x_1x_2\dots$,若$x$满足:\\
        $\exists N>0,x_N=1,x_j\neq 1(j<N),x_j=2(j>N),j=1,\dots$\\
        则将$x$改写为$0.x_1\dots x_{N-1}2$\\
        于是$C=\left\{x\in[0,1]:\text{在上述改写后,x的3进小数中没有1}\right\}$\\
        做映射$f:0.x_1x_2,\dots\mapsto \sum_0^{\infty}\frac{x_j}{2}*2^{-j}$\\
        右端为$[0,1]$的2进小数,于是$f$为C到[0,1]的满射,$card(C)\geq \mathcal{C}$\\
        又$C\subset[0,1]$,故$card(C)\leq\mathcal{C}$\\
        综上,$card(C)=\mathcal{C}$\\
        a.首先$C=[0,1]\cap F^c$为有界闭集,故是紧集\\
        C无处稠密$\Leftrightarrow \overline{C}^{\circ}=\phi\Leftrightarrow C^{\circ}=\phi$\\
        否则设$x\in C^{\circ}$,$\exists N\in\mathbb{N},(x-3^{-N},x+3^{-N})\subset C,\exists k=0,1,\dots,3^N-1\\
        (k3^{-N}+3^{-(N+1)},k3^{-N}+2*3^{-(N+1)})\subset(k3^{-N},(k+1)3^{-N})\\
        \subset(x-3^{-N},x+3^{-N})$从而$C\cap F\neq\phi$矛盾\\
        若C有连通部分是区间,则同上讨论知$C\cap F\neq\phi$矛盾\\
        设$x\in C$,则$x=0.x_1x_2\dots$,其中$x_j=0,2$\\
        $\overline{x}_N:=0.x_1x_2\dots x_N$于是$\overline{x}_N\in C$且$|x-\overline{x}_N|\leq 2*3^{-N}$\\
        这说明了$\forall x\in C$是$C$的聚点,也即$C$无孤立点
    \end{proof}
\end{proposition}

\end{document}
