\documentclass[12pt, a4paper, oneside]{ctexbook}
\usepackage{amsmath, amsthm, amssymb, bm, graphicx, hyperref, mathrsfs}

\title{{\Huge{\textbf{尸变函数抄书笔记}}}\\———主要参考Folland}
\author{Joker}
\date{\today}
\linespread{1.5}
\newtheorem{theorem}{定理}[section]
\newtheorem{definition}[theorem]{定义}
\newtheorem{lemma}[theorem]{引理}
\newtheorem{corollary}[theorem]{推论}
\newtheorem{example}[theorem]{例}
\newtheorem{proposition}[theorem]{命题}

\begin{document}
\begin{definition}
    对于非空集$X,Y$\\
    若存在单射$f:X\rightarrow Y$,则称$card(X)\leq card(Y)$\\
    若存在满射$f:X\rightarrow Y$,则称$card(X)\geq card(Y)$\\
    若存在双射$f:X\rightarrow Y$,则称$card(X)= card(Y)$\\
    若$card(X)\leq card(Y)$但$card(X)= card(Y)$不成立,则记作$card(X)<card(Y)$\\
    若$card(X)\geq card(Y)$但$card(X)= card(Y)$不成立,则记作$card(X)>card(Y)$\\
    对所有$X\neq\phi$,约定$card(X)>card(\phi)$,$card(\phi)<card(X)$
\end{definition}
\begin{corollary}
    $card(X)\leq card(Y),\ card(Y)\leq card(Z)\Rightarrow card(X)\leq card(Z)$
    \begin{proof}
        $f:X\rightarrow Y,\ g:Y\rightarrow Z$为单射,则\\
        $g\circ f:X\rightarrow Z$也是单射,故\\
        $card(X)\leq card(Z)$
    \end{proof}
\end{corollary}
\begin{proposition}
    $card(X)\leq card(Y)\Leftrightarrow card(Y)\geq card(X)$
    \begin{proof}
        设$f:X\rightarrow Y$为单射。取定$x_0\in X$,由于$f$为单射,$\forall y\in f(X)$,\\
        $\exists ! x\in X,f(x)=y$,于是可定义$g:Y\rightarrow X$:\\
        $g(y)=\begin{cases}
            f^{-1}(y)& \text{ if } y\in f(X) \\
            x_0& \text{ if } y\notin f(X)
          \end{cases}$\\
        从而$g:Y\rightarrow X$为满射\\[0.5cm]
        设$g:Y\rightarrow X$为满射,则$\forall x\in X,g^{-1}(\left\{x\right\})\neq\phi$\\
        且$x_1\neq x_2\Rightarrow g^{-1}(\left\{x_1\right\})\cap g^{-1}(\left\{x_2\right\})=\phi$\\
        由选择公理,$f\in \prod_{x\in X}g^{-1}(\left\{x\right\})$即为$X\rightarrow Y$的单射       
    \end{proof}
\end{proposition}
\begin{proposition}
    对任意集合$X,Y$,$card(X)\leq card(Y)$或$card(Y)\leq card(X)$成立
    \begin{proof}
        设$\mathcal{J}$为$X$的子集到$Y$的单射之集,即\\
        $\mathcal{J}=\left\{f:X\supset E\rightarrow Y\text{为单射}\right\}$\\
        $\forall f\in\mathcal{J},f\subset X\times Y$,于是$(\mathcal{J},\subset)$为一偏序集\\
        现在验证该偏序集适合Zorn's Lemma的条件\\
        $\forall f\in \mathcal{J},f:X\supset D_{f}\rightarrow Y$为单射\\
        设$\mathcal{I}$为$\mathcal{J}$的一个全序子集,$f_{\mathcal{I}}:=\bigcup_{f\in\mathcal{I}}f$\\
        由于$\mathcal{I}$为全序集,$\left\{D_f:f\in\mathcal{I}\right\}$在包含关系下也为全序集\\
        1.验证$f_{\mathcal{I}}\in\mathcal{J}$\\
        $D_{\mathcal{I}}:=\bigcup_{f\in\mathcal{I}}D_f$,$\forall x\in D_{\mathcal{I}}$,$\exists f\in\mathcal{I},x\in D_f$\\
        $\exists y\in Y,(x,y)\in f\subset f_{\mathcal{I}}$\\
        若存在$x_1\neq  x_2$,$f_{\mathcal{I}}(x_1)=f_{\mathcal{I}}(x_2)=y$\\
        则$(x_1,y)\in f_1,(x_2,y)\in f_2$,不妨设$f_1\subset f_2$\\
        于是有$x_1\neq x_2\rightarrow f_2(x_1)=f_2(x_2)$,这与$f_2\in\mathcal{J}$矛盾\\
        综上,$f_{\mathcal{I}}\in\mathcal{J}$\\
        2.验证$f_{\mathcal{I}}$为$\mathcal{I}$的上界,这由$f_{\mathcal{I}}$的定义是明显的\\
        综上,$\mathcal{J}$适合Zorn'Lemma的条件,从而有极大元$f_{\mathcal{J}}$\\
        设$f_{\mathcal{J}}:X\supset D_{\mathcal{J}}\rightarrow Y$,我们断言,$D_{\mathcal{J}}=X$或$f(D_{\mathcal{J}})=Y$成立\\
        否则,存在$x_0\in X\setminus D_{\mathcal{J}},y_0\in Y\setminus f(D_{\mathcal{J}})$\\
        $\tilde{f}_{\mathcal{J}}:=f_{\mathcal{J}}\cup\left\{(x_0,y_0)\right\}$\\
        容易验证$\tilde{f}_{\mathcal{J}}\in\mathcal{J}$且$f_{\mathcal{J}}\subsetneq \tilde{f}_{\mathcal{J}}$,这与$f_{\mathcal{J}}$是极大元矛盾\\
        若$D_{\mathcal{J}}=X$则$f_{\mathcal{J}}:X\rightarrow Y$为单射,$card(X)\leq card(Y)$\\
        若$f(D_{\mathcal{J}})=Y$,则$f_{\mathcal{J}}^{-1}:Y\rightarrow D_{\mathcal{J}}\subset X$为单射,$card(Y)\leq card(X)$
    \end{proof}
\end{proposition}
\begin{theorem}
    $The\ Sch\ddot{o}der-Bernstein\ Theorem$\\
    $card(X)\leq card(Y),card(Y)\leq card(X)\Rightarrow card(X)=card(Y)$
    \begin{proof}
        设$f:X\rightarrow Y$, $g:Y\rightarrow X$均为单射\\
        我们按照以下法则将$X$中的元素分为3类,\\
        1.对$\forall n\geq 0,\ g^{-1}\circ(f^{-1}\circ g^{-1})^{n}(x)\in Y,(f^{-1}\circ g^{-1})^{n+1}(x)\in X$\\
        则称$x\in X_{\infty}$\\
        2.$\exists n\geq 0,\ g^{-1}\circ(f^{-1}\circ g^{-1})^{n}(x)\notin Y$即$(f^{-1}\circ g^{-1})^{n}(x)\in X\setminus g(Y)$\\
        则称$x\in X_{X}$\\
        3.$\exists n\geq 0,\ (f^{-1}\circ g^{-1})^{n+1}(x)\notin X$即$g^{-1}\circ(f^{-1}\circ g^{-1})^{n}(x)\in Y\setminus f(X)$\\
        则称$x\in X_{Y}$\\
        容易验证,$X_{\infty},X_X,X_Y$是不交的,类似的有$Y_{\infty},Y_X,Y_Y:$\\
        1.对$\forall n\geq 0,\ f^{-1}\circ(g^{-1}\circ f^{-1})^{n}(y)\in X,(g^{-1}\circ f^{-1})^{n+1}(y)\in Y$\\
        则称$y\in Y_{\infty}$\\
        2.$\exists n\geq 0,\ f^{-1}\circ(g^{-1}\circ f^{-1})^{n}(y)\notin X$即$(g^{-1}\circ f^{-1})^{n}(y)\in Y\setminus f(X)$\\
        则称$y\in Y_{Y}$\\
        3.$\exists n\geq 0,\ (g^{-1}\circ f^{-1})^{n+1}(y)\notin Y$即$f^{-1}\circ(g^{-1}\circ f^{-1})^{n}(y)\in X\setminus g(Y)$\\
        则称$y\in Y_{X}$\\
        现在验证:$f(X_{\infty})=Y_{\infty},\ f(X_X)=Y_X,\ g(Y_Y)=X_Y$\\
        $x\in X_{\infty}\Rightarrow \forall n\geq 0,\ g^{-1}\circ(f^{-1}\circ g^{-1})^{n}(x)\in Y,(f^{-1}\circ g^{-1})^{n+1}(x)\in X$\\
        $\Rightarrow \forall n\geq 0,\ (g^{-1}\circ f^{-1})^{n+1}(f(x))\in X,f^{-1}\circ(g^{-1}\circ f^{-1})^{n}(y)\in Y$\\
        $\Rightarrow f(X)\subset Y_{\infty}$\\
        $y\in Y_{\infty}\Rightarrow x=f^{-1}(y)\in X$\\
        $\Rightarrow \forall n\geq 0,\ g^{-1}\circ(f^{-1}\circ g^{-1})^{n}(x)=(g^{-1}\circ f^{-1})^{n+1}(y)\in Y,$\\
        $(f^{-1}\circ g^{-1})^{n+1}(x)=f^{-1}\circ (g^{-1}\circ f^{-1})^{n+1}(x)\in X$\\
        $\Rightarrow \exists x\in X_{\infty},f(x)=y$\\
        $\Rightarrow f(X_{\infty})\subset Y_{\infty}$\\
        即$f(X_{\infty})=Y_{\infty}$,\\
        $x\in X_{X}\Rightarrow \exists n\geq 0,\ g^{-1}\circ(f^{-1}\circ g^{-1})^{n}(x)=(g^{-1}\circ f^{-1})^{n+1}(f(x))\notin Y$\\
        $\Rightarrow f(x)\in Y_X$\\
        $\Rightarrow f(X_X)\subset Y_X$\\
        $y\in Y_X\Rightarrow \exists n\geq 0,\ (g^{-1}\circ f^{-1})^{n+1}(y)\notin Y,f^{-1}(y)=x\in X$\\
        $\Rightarrow \exists n\geq 0,\ g^{-1}\circ(f^{-1}\circ g^{-1})^{n}(x)=(g^{-1}\circ f^{-1})^{n+1}(y)\notin Y$\\
        $\Rightarrow x=f^{-1}(y)\in X_X$\\
        $\Rightarrow Y_X\subset f(X_X)$\\
        即$f(X_X)=Y_X$\\
        同理,有$g(Y_Y)=X_Y$\\
        由于$f,g$均为单射,$f_{|X_{\infty}},f_{|X_X},g_{|Y_Y}$也是单射,从上面的讨论知,他们也是满射,从而为双射\\
        定义$h:X\rightarrow Y:$\\
        $h(x)=\begin{cases}
            f(x)& \text{ if } x\in X_{\infty}\cup X_X \\
            g^{-1}(x)& \text{ if } x\in X_Y
          \end{cases} $\\
        于是$h$为$X$到$Y$的双射,即$card(X)=card(Y)$
    \end{proof}
\end{theorem}
\begin{corollary}
    $card(X)=card(Y),card(Y)=card(Z)\Rightarrow card(X)=card(Z)$
    \begin{proof}
        $card(X)\leq card(Y),card(Y)\leq card(Z)\Rightarrow card(X)\leq card(Z)$\\
        同理$card(Z)\leq card(X)$\\
        于是$card(X)=card(Z)$
    \end{proof}
\end{corollary}
\begin{proposition}
    对任意集合$X,card(X)<card(\mathcal{P}(X))$
    \begin{proof}
        首先,$f:x\mapsto \left\{x\right\}$是$X$到$\mathcal{P}(X)$的单射,故$card(X)\leq card(\mathcal{P}(X))$\\
        设$g:X\rightarrow \mathcal{P}(X)$,我们来证明$g$不可能是满射,\\
        令$Y=\left\{x\in X:x\notin g(x)\right\}$\\
        若$Y\in g(X)$,即$\exists x_0\in X,g(x_0)=Y$\\
        那么$x_0\in Y\Rightarrow x_0\notin g(x_0)\Rightarrow x_0\notin Y$,矛盾\\
        $x_0\notin Y\Rightarrow x_0\in g(x_0)\Rightarrow x_0\in Y$,矛盾\\
        故不存在$x_0\in X,g(x_0)=Y$\\
        综上,$card(X)<card(\mathcal{P}(X))$
    \end{proof}
\end{proposition}
\begin{definition}
    若$card(X)\leq card(\mathbb{N})$,则称$X$是可数的
\end{definition}
\begin{proposition}
    a.可数集合的有限笛卡尔积可数\\
    b.可数集合的可数并可数\\
    c.可数无穷集与自然数集等势
    \begin{proof}
        a.设$X,Y$为可数集,$f:X\rightarrow \mathbb{N},\ g:Y\rightarrow\mathbb{N}$为单射\\
        则$f\times g:(x,y)\mapsto (f(x),g(y))$为单射,从而$card(X\times Y)\leq card(\mathbb{N}^2)$\\
        现在证明: $card(\mathbb{N})=card(\mathbb{N^2})$\\
        构造$\mathbb{N}^2$到$\mathbb{N}$的双射:$f:(i,j)\mapsto i+\sum_{n=1}^{i+j-2}n$\\[0.5cm]
        b.设$A,X_{\alpha}(\alpha\in A)$均为可数集,$\forall \alpha\in A$,存在 $f_{\alpha}:\mathbb{N}\rightarrow X_{\alpha}$为满射\\
        于是$f:\mathbb{N}\times A\rightarrow \bigcup_{\alpha\in A}X_{\alpha},\ f(n,\alpha)=f_{\alpha}(n)$为满射\\
        从而$card(\bigcup_{\alpha\in A}X_{\alpha})\leq card(A\times \mathbb{N})\leq card(\mathbb{N})$\\[0.5cm]
        c.设$X$为无穷集合,且$card(X)\leq card(\mathbb{N})$\\
        $f:X\rightarrow \mathbb{N}$为单射,则$f:X\rightarrow f(X)\subset \mathbb{N}$为双射\\
        $card(X)=card(f(X))\leq card(\mathbb{N})$\\
        现在记$Y=f(X)$为$\mathbb{N}$的无穷子集\\
        定义$g(1)=min\ Y$\\
        递归定义$g(n)=min\ Y\setminus\left\{g(1),g(2),\dots g(n-1)\right\}$\\
        于是$g:N\rightarrow Y$为双射\\
        $card(Y)=card(\mathbb{N})$\\
        从而$card(X)=card(\mathbb{N})$
    \end{proof}
\end{proposition}
\begin{corollary}
    $card(\mathbb{Z})=card(\mathbb{Q})=card(\mathbb{N})$
\end{corollary}
\begin{definition}
    若$card(X)=card(\mathbb{R})$,则记作$card(X)=\mathcal{C}$
\end{definition}
\begin{proposition}
    $card(X)=card(Y)\Rightarrow card(\mathcal{P}(X))=card(\mathcal{P}(Y))$
    \begin{proof}
        设$f:X\rightarrow Y$为双射\\
        $\tilde{f}:\mathcal{P}(X)\rightarrow\mathcal{P}(Y),\ \tilde{f}(A)=f(A)$给出$\mathcal{P}(X)$到$\mathcal{P}(Y)$的双射
    \end{proof}
\end{proposition}
\begin{proposition}
    $card(\mathcal{P}(\mathbb{N}))=\mathcal{C}$
    \begin{proof}
        对于$A\in\mathcal{P}(N)$,定义$f(A)=\begin{cases}
            \sum_{n\in A}2^{-n}& \text{ if } \mathbb{N}\setminus A\text{是无穷集}  \\
            1+\sum_{n\in A}2^{-n}& \text{ if } \mathbb{N}\setminus A\text{是有限集}
          \end{cases}$\\
        $f:\mathcal{P}(\mathbb{N})\rightarrow \mathbb{R}$是单射(?),故$card(\mathcal{P}(\mathbb{N}))\leq card(\mathbb{R})$\\
        定义$g:\mathcal{P}(\mathbb{Z})\rightarrow \mathbb{R},\ g(A)=\begin{cases}
            log(\sum_{n\in A}2^{-n})& \text{ if } A\text{有下界} \\
            0& \text{ if } A\text{无下界}
          \end{cases}$\\
        $card(\mathcal{P}(\mathbb{N}))=card(\mathcal{P}(\mathbb{Z}))\geq card(\mathbb{R})$\\
        综上,$card(\mathcal{P}(\mathbb{N}))=\mathcal{C}$
    \end{proof}
\end{proposition}
\begin{corollary}
    若$card(X)\geq\mathcal{C}$,则$X$是不可数的
    \begin{proof}
        若$X$可数,$\mathcal{C}=card(\mathcal{P}(\mathbb{N}))>card(\mathbb{N})\geq card(X)\geq \mathcal{C}$,矛盾
    \end{proof}
\end{corollary}
\begin{example}
    连续统假设:$card(X)<\mathcal{C}$则$X$是可数的
\end{example}
\begin{proposition}
    a.$card(X)\leq\mathcal{C},\ card(Y)\leq\mathcal{C}$,则$card(X\times Y)\leq\mathcal{C}$\\
    b.$card(A)\leq\mathcal{C},card(X_{\alpha})\leq\mathcal{C}\Rightarrow card(\bigcup_{\alpha\in A}X_{\alpha})\leq\mathcal{C}$
    \begin{proof}
        a.$\exists f:X\rightarrow \mathcal{P}(\mathbb{N}),\ g:Y\rightarrow \mathcal{P}(\mathbb{N})$为单射\\
        $f\times g:X\times Y\rightarrow (\mathcal{P}(\mathbb{N}))^2,\ (f\times g)(x,y)=(f(x),g(y))$为单射\\
        于是$card(X\times Y)\leq card((\mathcal{P}(\mathbb{N}))^2)$\\
        现在证明$card((\mathcal{P}(\mathbb{N}))^2)=card(\mathcal{P}(\mathbb{N})):$\\
        定义$\phi,\psi :\mathbb{N}\rightarrow\mathbb{N}$\\
        $\phi(n)=2n,\ \psi(n)=2n-1$\\
        定义$f:(\mathcal{P}(\mathbb{N}))^2\rightarrow \mathcal{P}(\mathbb{N}),\ f(A,B)=\phi(A)\cup\psi(B)$\\
        则$f$为双射\\
        综上$card(X\times Y)\leq card((\mathcal{P}(\mathbb{N}))^2)=\mathcal{C}$\\[0.5cm]
        b.$\forall\alpha\in A,\ \exists f_{\alpha}:\mathcal{P}(\mathbb{N})\rightarrow X_{\alpha}$为满射\\
        定义$f:A\times\mathcal{P}(\mathbb{N})\rightarrow \bigcup_{\alpha\in A}X_{\alpha},\ f(\alpha,E)=f_{\alpha}(E)$为满射\\
        故$card(\bigcup_{\alpha\in A}X_{\alpha})\leq card(A\times\mathcal{P}(N))\leq\mathcal{C}$
    \end{proof}
\end{proposition}
\end{document}