\documentclass[12pt, a4paper, oneside]{ctexbook}
\usepackage{amsmath, amsthm, amssymb, bm, graphicx, hyperref, mathrsfs}

\title{{\Huge{\textbf{尸变函数抄书笔记}}}\\———主要参考Folland}
\author{Joker}
\date{\today}
\linespread{1.5}
\newtheorem{theorem}{定理}[section]
\newtheorem{definition}[theorem]{定义}
\newtheorem{lemma}[theorem]{引理}
\newtheorem{corollary}[theorem]{推论}
\newtheorem{example}[theorem]{例}
\newtheorem{proposition}[theorem]{命题}

\begin{document}
\begin{definition}
    设$X$为一个配备$\sigma$-代数$\mathcal{M}$的集合,$\mathcal{M}$上的测度是$\mathcal{M}\rightarrow[0,+\infty]$的满足以下性质的函数:\\
    1. $\mu(\phi)=0$;\\
    2. 若$\left\{E_j\right\}_1^{\infty}\subset\mathcal{M}$为不交的集和,那么$\mu(\bigcup_1^{\infty}E_j)=\sum_{1}^{\infty}\mu(E_j)$\\
    $(X,\mathcal{M})$称为可测空间,$\mathcal{M}$中的集合称为可测集\\
    若$\mu$是$(X,\mathcal{M})$上的测度,$(X,\mathcal{M},\mu)$称为测度空间
\end{definition}
    设$(X,\mathcal{M},\mu)$为测度空间,\\
    若$\mu(X)<+\infty(\Leftrightarrow\mu(E)<+\infty,\forall E\in\mathcal{M})$则称$\mu$为有限的,\\
    若$X=\bigcup_1^{\infty}E_j$其中$E_j\in\mathcal{M}$且$\mu(E_j)<+\infty$则称$\mu$是$\sigma$-有限的,\\
    若$E=\bigcup_1^{\infty}E_j$其中$E_j\in\mathcal{M}$且$\mu(E_j)<+\infty$则称$E$是关于$\mu\ \sigma$-有限的\\
    若对于每个$E$,$\mu(E)=+\infty\Rightarrow\exists F\in\mathcal{M},F\subset E$,且$0<\mu(F)<+\infty$\\
    则称$\mu$是semifinite
\begin{example}
    设$X$为一非空集合,$\mathcal{M}=\mathcal{P}(X)$,$f$为任意$X\rightarrow[0,+\infty]$的函数,
    则$f$可以诱导一个测度:$\mu(E)=\sum_{x\in E}f(x)$\\
    其中$\sum_{x\in E}f(x):=\sup\left\{\sum_{x\in F}f(x):F\subset E,F\text{为有限集}\right\}$\\
    $\mu$为semifinite的当且仅当$f(x)<+\infty,\forall x\in X$\\
    $\mu$为$\sigma$-有限的当且仅当$\mu$为semifinite并且$\left\{x:f(x)>0\right\}$是可数集\\
    若$f(x)\equiv 1$,$\mu$称为计数测度\\
    对某个$x_0\in X$,$f(x_0)=1,\forall x\neq x_0,f(x)=0$则$\mu$称为point mass或Dirac measure at $x_0$
\end{example}
\begin{example}
    设$X$为一不可数集,$\mathcal{M}$为其可数或补集可数的子集构成的$\sigma$-代数,
    则$\mu$定义为:$\mu(E)=\begin{cases}
        0& \text{ if } E\text{为可数集} \\
        1& \text{ if } E^c\text{为可数集}
      \end{cases}$是$(X,\mathcal{M})$上的测度
\end{example}
\begin{example}
    令$X$为一无穷集,$\mathcal{M}=\mathcal{P}(X)$定义$\mu(E)=\begin{cases}
        0& \text{ if } E\text{为有限集} \\
        +\infty& \text{ if } E\text{为无穷集}
      \end{cases}$\\
      则$\mu$为有限可加的,但不是测度
\end{example}
\begin{theorem}
    令$(X,\mathcal{M},\mu)$为一测度空间\\
    a.(Monotonicity) $E,F\in\mathcal{M},E\subset F\Rightarrow\mu(E)\leq\mu(F)$\\
    b.(Subadditivity) $\left\{E_j\right\}_1^{\infty}\subset\mathcal{M}\Rightarrow\mu(\bigcup_1^{\infty}E_j)\leq\sum_1^{\infty}\mu(E_j)$\\
    c.(Continuity from below) $\left\{E_j\right\}_1^{\infty}\subset\mathcal{M},E_1\subset E_2\subset\dots\\
    \Rightarrow\mu(\bigcup_1^{\infty}E_j)=\lim_{j \to \infty}\mu(E_j) $\\
    d.(Continuity from above) $\left\{E_j\right\}_1^{\infty}\subset\mathcal{M},E_1\supset E_2\supset\dots$且$\mu(E_1)<\infty$\\
    则$\mu(\bigcap_1^{\infty}E_j)=\lim_{j \to \infty}\mu(E_j)$
    \begin{proof}
        a. $\mu(F)=\mu(E)+\mu(F\setminus E)\geq\mu(E)$\\
        b. 令$F_1=E_1,k>1,F_k=E_k\setminus(\bigcup_1^{k-1}E_j),\\
        \bigcup_1^{\infty}F_k=\bigcup_1^{\infty}E_k$且$F_k\in\mathcal{M}$是不交的\\
        $F_k\subset E_k$,由a.知$\mu(F_k)\leq\mu(E_k)$\\
        $\mu(\bigcup_1^{\infty}E_k)=\mu(\bigcup_1^{\infty}F_k)=\sum_1^{\infty}\mu(F_k)\leq\sum_1^{\infty}\mu(E_k)$\\
        c. $\mu(\bigcup_1^{\infty}E_k)=\mu(\bigcup_1^{\infty}(E_k\setminus E_{k-1}))$其中$E_0=\phi$\\
        $=\sum_1^{\infty}\mu(E_k\setminus E_{k-1})=\lim_{n \to \infty}\sum_1^n\mu(E_k\setminus E_{k-1})=\lim_{n \to \infty}\mu(E_n)$\\
        d. 令$F_k=E_1\setminus E_k$则$F_1\subset F_2\subset\dots$,\\
        $\bigcap_1^{\infty}E_k=\bigcap_1^{\infty}(E_1\setminus F_k)=E_1\setminus\bigcup_1^{\infty}F_k$\\
        于是$\mu(E_1)=\mu(\bigcup_1^{\infty}F_k)+\mu(\bigcap_1^{\infty}E_k)$\\
        $=\lim_{n \to \infty}\mu(F_n)+\mu(\bigcap_1^{\infty}E_k)$\\
        $=\mu(E_1)-\lim_{n \to \infty}\mu(E_n)+\mu(\bigcap_1^{\infty}E_k)$\\
        其中$\mu(E_1)<+\infty$,故$\mu(\bigcap_1^{\infty}E_k)=\lim_{n \to \infty}\mu(E_n)$
    \end{proof}
    注意d.中$\mu(E_1)<+\infty$可以改为对某个$j$成立$\mu(E_j)<+\infty$
\end{theorem}
设$(X,\mathcal{M},\mu)$为一测度空间,$E\in\mathcal{M}$满足$\mu(E)=0$,则称$E$为零测集,由
subadditivity,零测集的任意可数并为零测集,若一个关于$x$的命题在一个零测集外成立,则称其
几乎处处成立(a.e.),更具体的,称为$\mu$-零测集或$\mu$-几乎处处\\
测度$\mu$称为完全的,若其定义域包含所有零测集的子集
\begin{theorem}
    设$(X,\mathcal{M},\mu)$为一测度空间,令$\mathcal{N}=\left\{N\in\mathcal{M}:\mu(N)=0\right\},\overline{\mathcal{M}}$
    $=\left\{E\cup F:E\in\mathcal{M},F\subset N,\text{对某个}N\in\mathcal{N}\right\}$那么$\overline{\mathcal{M}}$是$\sigma$-代数,并且存在唯一的
    $\overline{\mu}$是$\mu$在$\overline{\mathcal{M}}$上的扩张,并且$\overline{\mu}$是完全的
    \begin{proof}
        首先证明$\overline{\mathcal{M}}$是$\sigma$-代数:\\
        由于$\mathcal{M}$和$\mathcal{N}$均对可数并封闭,故$\overline{\mathcal{M}}$也对可数并封闭\\
        对于$E\in\mathcal{M}$,$F\subset N\in\mathcal{N}$,$E\cup F\in\overline{M}$,不妨设$E\cap N=\phi$,否则可用$F\setminus E,N\setminus E$
        代替$F,N$,于是$(E\cup N)\cap(N^c\cup F)=(E\cap N^c)\cup(E\cap F)\cup(N\cap N^c)\cup(N\cap F)$
        $=E\cup(E\cap F)\cup\phi\cup F=E\cup F$\\
        故$(E\cup F)^c=(E\cup N)^c\cup(N\setminus F)$其中$(E\cup F)^c\in\mathcal{M}$,$N\setminus F\subset N$\\
        故$(E\cup F)^c\in\overline{\mathcal{M}}$,从而$\overline{\mathcal{M}}$是$\sigma$-代数\\
        对于$E\cup F\in\overline{\mathcal{M}}$,定义$\overline{\mu}(E\cup F)=\mu(E)$\\
        现在验证该定义是良定的:\\
        $E_1\cup F_1=E_2\cup F_2\Rightarrow E_1\subset E_2\cup F_2\subset E_2\cup N_2$\\
        于是$\mu(E_1)\leq\mu(E_2)+\mu(N_2)=\mu(E_2)$\\
        同理$\mu(E_2)\leq\mu(E_1)$\\
        即$\overline{\mu}(E_1\cup F_1)=\overline{\mu}(E_2\cup F_2)$\\
        现在验证$\overline{\mu}$是测度:\\
        1. $\overline{\mu}(\phi)=\overline{\mu}(\phi\cup\phi)=\mu(\phi)=0$\\
        2. 设$\left\{E_j\cup F_j\right\}_1^{\infty}$为不交的集列,其中$E_j\in\mathcal{M},F_j\subset N_j\in\mathcal{N}$\\
        容易验证$E_j$是不交的,$\bigcup_1^{\infty}F_j\subset\bigcup_1^{\infty}N_j\in\mathcal{N}$\\
        于是$\overline{\mu}(\bigcup_1^{\infty}(E_j\cup F_j))=\overline{\mu}((\bigcup_1^{\infty}E_j)\bigcup(\bigcup_1^{\infty}F_j))=\mu(\bigcup_1^{\infty}E_j)=$\\
        $\sum_1^{\infty}\mu(E_j)=\sum_1^{\infty}\overline{\mu}(E_j\cup F_j)$\\
        下面验证该测度是完全的:\\
        设$E\in\mathcal{M},F\subset N\in\mathcal{N}$\\
        $\overline{\mu}(E\cup F)=0\Rightarrow\mu(E)=0\Rightarrow E\in\mathcal{N}$ 于是$E\in\mathcal{N},N\cup E\in\mathcal{N}$\\
        $\forall U\subset E\cup F,U=\phi\cup U$,其中$U\subset E\cup F\subset E\cup N\in\mathcal{N}$\\
        即$U\in\overline{\mathcal{M}}$\\
        现在验证这样的扩张是唯一的:\\
        设$\widetilde{\mu}$是另一个扩张,则其在$\mathcal{M}$上的限制应与$\mu$相同\\
        即$\forall E\in\mathcal{M},\widetilde{\mu}(E)=\mu(E)$\\
        于是对于$E\cup F\in\overline{\mathcal{M}}$,其中$E\in\mathcal{M},F\subset N\in\mathcal{N}$\\
        $\widetilde{\mu}(E\cup F)\leq\widetilde{\mu}(E)+\widetilde{\mu}(F)\leq\widetilde{\mu}(E)+\widetilde{\mu}(N)=\mu(E)+\mu(N)=\mu(E)=\overline{\mu}(E\cup F)$
        $\overline{\mu}(E\cup F)=\mu(E)=\widetilde{\mu}(E)\leq\widetilde{\mu}(E\cup F)$\\
        即$\overline{\mu}=\widetilde{\mu}$
    \end{proof}
\end{theorem}
\end{document}